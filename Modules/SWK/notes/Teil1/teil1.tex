\section{Teil1}

\subsection{Projektmanagement}

Entwicklungsphasen:\\
\\
Weil unterteilen den Weg von der Idee/ dem Problem hin zum Betrieb in Phasen.

\begin{itemize}
    \item Analyse
    \item Design
    \item Umsetzung
    \item Test
\end{itemize}

Dazu werden wir uns Methodik, Metriken und Werkzeuge anschauen.
Das Querschnittstheme dieser Sektion ist das Projektmanagement, später werden wir noch Methodik im Vorgehen genauer 
betrachten.\\
\\ \\
Anforderungen an Methoden:\\
\\
Die Methode sollte korrekt (Qualität), im Budget (Kosten) und termingerecht (Termin) umgesetzt werden können.

\subsection{Projektplanung}

\subsubsection{Wasserfallmodell}

Beim Wasserfallmodell werden Phasen sukzessiv abgearbeitet, ohne Rückkanten.
\begin{enumerate}
    \item Anforderung
    \item Entwurf
    \item Implementierung
    \item Überprüfung
    \item Wartung
\end{enumerate}

\subsubsection{V-Modell}

Das V-Modell beschreibt ein linear Steigende und dann linear fallendes Verhältnis von Zeit und umfang/ Detaillierung vom
Entwurf.
Zeitlich mittig und mit der größten Detaillierung steht der software-Entwurf.
Anforderungen, System-Architektur, System-Entwurf, sowie Software Architektur gehen disem zuvor und
Unit-Tests, Integrations-Tests, Sytem-Integration, sowie Abnahme und Nutzung folgt offensichtlich.

\subsubsection{Iterative Modell}

\subsubsection{SCRUM}

Productowner erstellt den Backlog (Priorisierte Wunschliste), im Sprint Planning wird ein Teil der höchsten 
Prioritäten genommen.
Der Sprint (30 Tage nach VL) selbst bezeichnet die Impelementierung.
Dabei wird im Daily SCRUM ein meeting über Forstschritte und Probleme gehalten.
Es Folgt zum Schluss eine Retrospektive zum Prozess.

\subsubsection{Vergleich der Modellvarianten}
\begin{tabular}{l|llll}
    Kriterium                           & Wasserfall            & Spiral        & Scrum         & Simultan Engineering\\
    \hline
    Ablauf                              & sequentiell           & iterativ      & iterativ      & parallel\\
    \hline
    Phasentrennung                      & ausgeprägt            & schwach       & schwach       & fehlt\\
    \hline
    Durchlaufzeit                       & lang                  & lang          & kurz          & kurz\\
    \hline
    Feststellung von Fehlern            & spät                  & früher        & früh          & spät\\
    \hline
    Aufwand Planung/ Kommunikation      & gering                & mittel        & niedrig/hoch  & hoch\\
\end{tabular}

\subsubsection{Ereignis Knoten Netz}

Das Ereignis Knoten Netz ist ein Graph, der die Abhängigkeiten von Aufgaben zueinander darstellt.
Kann ein Ereignis nicht geschehen, ohne dass zuvor ein anderes abgelaufen ist, so hat dessen Knoten im Graphen eine
eingehende Kante von dem Ereignis, das zuvor geschehen sein muss.
\\ \\
\textit{Crititcal-Path-Method}\\
Die Crititcal-Path-Method stützt sich auf das Ereignis Knoten Netz, bzw. erweitert dessen Syntax und Semantik.
So wird in jedem Knoten die Ereignisnummer, der früheste - und späteste Ereignistermin und der Puffer eingetragen.
Die Kanten erhalten zudem noch die Dauer des davor liegenden Verfahrens als gewichtung, sowie eine ID.

\begin{array}[t]{ll}
    j&:$ Ereignisnummer$\\
    F&:$ frühester\ Ereignistermin$\\
    S&:$ spätester\ Ereignistermin$\\
    P&:$ Puffer$\\
    D&:$ Dauer$\\ 
\end{array}

\subsection{Schätzen, Messen und Steuern}

\subsubsection{Schätzen}

Das Aufwands-Aufrags-Dilemma behandelt das Problem, dass die Zeit die für einen Auftrag angegeben wird den Auftrag sichert.
Eine kurze angebene Auftagszeit sichert den Auftrag, birgt jedoch ein hohes Risiko, vice versa eine lange angegebene Auftragszeit.
\\ \\
Es gibt folgende Schätzmethoden:\\
\begin{tabular}{p{2cm}p{11cm}}
    Methode& Beispiel\\ \hline \hline
    Intuitiv& Raten\\ \hline
    Vergleich& mehr/ weniger als (unsicher, einfach)\\ \hline
    Kennzahlen& Einflussparameter $A = \sum_i c_i E_i$\\ \hline
    Zerlegen& Summe von Einzelschätzungen $A = \sum_i A_i$\\ \hline
    Skalieren& Dimensionen reduzieren und auf anschauliche Größen reduzieren\\ \hline
    Kombinieren& kombination von Methoden\\ \hline
    Gruppe& Gruppen schätzen besser als eine Person (Delphi Methode)\\ \hline
\end{tabular}
\\ \\ \\
\textit{Punktschätzung und Aufwand}

Es geht darum den Aufwand $E(x)$ und die Standardabweichung $S(x)$ aus gegebenen Parametern zu schätzen.
Allgemein gilt $$
    E(x) = \frac{a+r\cdot c+b}{2+r}, V(x) = \frac{(b-1)^2}{u^2}, S(x) = \frac{(b-a)}{u}
$$, wobei $a \leq c \leq b$
\\ \\
Spezialfälle sind:\\
\begin{array}[t]{ll}
    Parameter& Formeln\\
    a,b& E(x) = \frac{a+b}{2}, S(x) = \frac{b-a}{6}\\
    a,c,b& E(x) = \frac{a+4\cdot c+b}{6}, S(x) = \frac{b-a}{6}\\
\end{array}

\subsubsection{Reaktionsmöglichkeiten}

\textbf{1. Fall über Plan}

Zunächst werden geringe Zeitvorteile als zusätzliche Puffer verwendet.
Danach wird der größere Zeitvorteil zur Planrevidierung verwendet.
\\ \\
\textbf{2. Fall unter Plan (Plan vs. Realität)}

Falsche Realität/ Rückstand aufholen:\\
Überstunden, mehr Personal und/ oder bessere Arbeitsleistungen werden benötigt.
\\ \\
Falscher Plan/ Plan revidieren:\\
einmalige Verschiebung des Zeitplans bei einmaliger Fehlplanung, Dehnung des Gesamtplans bei systematischer Fehllanung.
\\ \\
\textbf{Umgang mit Abweichungen}

Kleine Abweichungen puffern, mittlere im Projektteam lösen, große mit dem Auftragsgeber.

\subsubsection{Risiko}

Das Risiko erreichnet sich aus der Risikowahrscheinlichkeit $p_i$ und dem Schadensausmaß $S_i$ $$
    R = \sum_i^N p_i S_i
$$

\subsection{Anforderungen}

\subsubsection{Typen und Eigenschaften}

\begin{tabular}{l|l}
    Funktional
            & Nicht Funktional\\
    \hline \hline
    Benutzer\\ 
        \tabitem Eingabe-/ Ausgabeverhalten\\
        \tabitem Was soll nicht passieren bei Benutzereingaben?\\
            & Unternehmensanforderungen\\
            &   \tabitem Anforderungen an den Entwicklungsprozess\\
    \\
    System\\
        \tabitem \textbf{Wie} realisiert das System Benutzeranforderungen?\\
            & Produktanforderungen\\
            &   \tabitem Performanz des Produkts\\
    \\
            & Externe Anforderungen\\
            &   \tabitem Rechtliche Rahmenbedingungen\\
\end{tabular}
\\ \\
Funktionale Anforderungen umfassen Leistungen, die das System abieten soll.
\\
Benutzeranforderungen beschreiben die Dienste, die eine Software leisten soll. 
Außerdem werden Rahmenbedingungen, unter denen die Software betrieben werden soll, definiert.
\\
Systemanforderungen definieren was wie implementiert werden soll.
\\ \\
Nicht funktionale Anforderungen umfassen Qualitätsanforderungen an Produkt und Prozess.
\\
Unternehmensanforderungen umfassen Anforderungen an den Entwicklungsprozess, sowie an die (Programmier-)Umgebung des
Produkts.
\\
Produktanforderungen umfassen Performanz, Zuverlässigkeit, Benutzbarkeit.
\\ 
Externe Anforderungen umfassen Regulatorische und ethische Anforderungen.

\subsection{Spezifikation von Anforderungen}

\subsubsection{Sophisten Satzschablonen}

SYSTEM - sollte/ muss/ wird/ ... - \_/ fähig sein/ die Möglichkeit bieten/ ... - OBJEKT - PROZESSWORT

\subsubsection{Formale Sprachen}

\subsection{Erheben und Management von Anforderungen}
