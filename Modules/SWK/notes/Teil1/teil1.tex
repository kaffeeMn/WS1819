\section{Projektmanagement}

Entwicklungsphasen:\\
\\
Weil unterteilen den Weg von der Idee/ dem Problem hin zum Betrieb in Phasen.

\begin{itemize}
    \item Analyse
    \item Design
    \item Umsetzung
    \item Test
\end{itemize}

Dazu werden wir uns Methodik, Metriken und Werkzeuge anschauen.
Das Querschnittstheme dieser Sektion ist das Projektmanagement, später werden wir noch Methodik im Vorgehen genauer 
betrachten.\\
\\ \\
Anforderungen an Methoden:\\
\\
Die Methode sollte korrekt (Qualität), im Budget (Kosten) und termingerecht (Termin) umgesetzt werden können.

\subsection{Wasserfallmodell}

Beim Wasserfallmodell werden Phasen sukzessiv abgearbeitet, ohne Rückkanten.
\begin{enumerate}
    \item Anforderung
    \item Entwurf
    \item Implementierung
    \item Überprüfung
    \item Wartung
\end{enumerate}

\subsection{V-Modell}

Das V-Modell beschreibt ein linear Steigende und dann linear fallendes Verhältnis von Zeit und umfang/ Detaillierung vom
Entwurf.
Zeitlich mittig und mit der größten Detaillierung steht der software-Entwurf.
Anforderungen, System-Architektur, System-Entwurf, sowie Software Architektur gehen disem zuvor und
Unit-Tests, Integrations-Tests, Sytem-Integration, sowie Abnahme und Nutzung folgt offensichtlich.

\subsection{Iterative Modell}

\subsection{SCRUM}

\subsection{Vergleich der Modellvarianten}
\begin{tabular}{llll}
    Kriterium                           & Wasserfall            & Spiral        & Scrum         & Simultan Engineering\\
    Ablauf                              & sequentiell           & iterativ      & iterativ      & parallel\\
    Phasentrennung                      & ausgeprägt            & schwach       & schwach       & fehlt\\
    Durchlaufzeit                       & lang                  & lang          & kurz          & kurz\\
    Feststellung von Fehlern            & spät                  & früher        & früh          & spät\\
    Aufwand Planung/ Kommunikation\\
\end{tabular}

\subsection{Ereignis Knoten Netz}

Das Ereignis Knoten Netz ist ein Graph, der die Abhängigkeiten von Aufgaben zueinander darstellt.
Kann ein Ereignis nicht geschehen, ohne dass zuvor ein anderes abgelaufen ist, so hat dessen Knoten im Graphen eine
eingehende Kante von dem Ereignis, das zuvor geschehen sein muss.

\subsubsection{Crititcal-Path-Method}

Die Crititcal-Path-Method stützt sich auf das Ereignis Knoten Netz, bzw. erweitert dessen Syntax und Semantik.
So wird in jedem Knoten die Ereignisnummer, der früheste - und späteste Ereignistermin und der Puffer eingetragen.
Die Kanten erhalten zudem noch die Dauer des davor liegenden Verfahrens als gewichtung, sowie eine ID.
