\documentclass{article}
\textwidth=6in
\hoffset=0in
\voffset=0in

\usepackage{dcolumn}
\usepackage{afterpage}
\usepackage{pgf}
\usepackage{tikz}
\usepackage{pdflscape}
\usetikzlibrary{arrows,automata}
\usepackage[latin1]{inputenc}
\usepackage{ngerman}
\usepackage[a4paper, total={6in, 8in}]{geometry}
\usepackage{amsmath}
\usepackage{amssymb}
\usepackage{stmaryrd}
\usepackage{graphicx}
\usepackage{tikz}
\usetikzlibrary{automata, arrows, fit, calc}
\usepackage{pifont}
\usepackage{amssymb}
\usepackage{gensymb}
\usepackage[ampersand]{easylist}


\usepackage{listings}

\newcommand{\gap}{\ \\ \\}

\setcounter{section}{2}

% needs to be updated
\author{Max Springenberg, 177792}
\title{SWK\\
       "Ubungsblatt 2\\
       Gruppe 2: OH12 / 1.056}
\setcounter{section}{2}


\begin{document}

\subsection{Umsetzung von Architekturmustern}
\subsubsection\ 
(i) Client-Server: Web-Anwendungen, wie z.B. unser moodle\\
Im moodle "ubernehmen Studenten die Client- und das System auf dem das moodle 
gehostet wird die Serverrolle. Die Anwendungslogik wird auf eine Menge von 
Studenten und Subsysteme von moodle verteilt.\\
\\
(ii) Peer-to-Peer: Bittorrent\\
Bittorrent ist eine Umsetzung der Peer-to-Peer Architektur.\\
Peers werden duch einen Server/ Broadcast gefunden. Diese werden dann zu 
seeders und leachers, also Kommunizieren untereinander und nicht mit dem Server.\\
\\
(iii) Pipe-and-Filter: Unix-Filter\\
Unix Pipes sind regul"are Ausdr"ucke, die compiliert werden. Dabei ist jeweils
ein Filter/ eine Funktion durch ein Pipe-Symbol `$|$` getrennt und dessen 
Ausgabe wird weitergereicht.\\
\\
(iv) Sharding: Data-Bases, wie z.B. oracle\\
Datenbanksysteme benutzen unter anderem Sharding (wenn gut implementiert in 
Kombination mit Load-Balancing) um Daten aufzuteilen und zu speichern.\\
\subsubsection\ 
Peer-to-Peer und Pipe-and-Filter sind ungeeignet.\\
\\
Grund daf"ur sind die in der Vorlesung genannten Nachteile. So kann bei 
    Peer-to-Peer der Server, durch den die Peers gefunden werden zu einem Engpass
    werden. Bezahlungen sollten schnell erfolgen k"onnen. Ferner sollten Konten
    eingerichtet und verwaltet werden k"onnen. Dies ist mit einer Peer-to-Peer
    Architektur nur erschwert m"oglich.\\
Pipe-and-Filter ist eine Architektur, die rein Funktional "uber das durchlaufen
    von Filtern funktioniert. Hierbei sind unter anderem Deadlocks m"oglich
    und das Verwalten, sowie I/O Operationen durch Filter sind ebenfalls nur
    erschwert m"oglich.\\

\subsection{Model-View-Whatever}
Wie grenzen sich die Muster untereinander ab?\\
Wie wird jeweils der Fluss von Daten zwischen den Komponenten realisiert?\\
Wie reagiert das System jeweils auf Benutzereingaben?\\
\gap
Bei einem Model-View-Controller Modell korelliert jede Aktion in einem View mit dem
    zu korrespondierenden Controller. Der Controller entscheidet welche View
    angezeigt wird.\\
Es gibt im Model-View-Controller mehrere Views, zu denen je ein Controller 
    existieren sollte. Der Kontroller nimmt jeweils die "Anderungen am Model vor,
    die f"ur den ihm zugeh"origen View relevant sind.\\
Beim Model-View-Controller Modell ist eine klare Top-Down Hierarchie vorhanden.\\
Der Benutzer bedient die GUI, bzw. interargiert mit der View, diese Teilt das
    dem zust"andigen Controller mit, der das Model editiert/ Updates vornimmt.
    Das Modell f"ullt dann den View.\\
\\
Bei einem Model-View-Presenter Modell funktioniert der Presenter "ahnlich zum
    Controller, jedoch verbindet der Presenter das Modell mit der View ein einer
    `dynamischeren` Weise. Jede Aktion im View geht direkt zum Presenter, der
    durch ein Interface mit dem View kommuniziert. Im Presenter ist ist also 
    s"amtliche Logik implementiert. Model und View werden so `dumm` wie 
    m"oglich gehalten.\\
Benutzereingaben werden also "uber die Logik des Presenters verarbeitet, der dann
    View und Modell w"ahrend der Laufzeit verbindet und "Anderungen vornimmt.\\
\\
Das Model-View-ViewModel Modell "uberl"asst der View mehr Eigenst"andigkeit in 
    ihrer Darstellung. Das Model benachrichtigt das ViewModel, dieses
    benachrichtigt die View lediglich, dass
    eine "Anderung passiert ist, die View "ubernimmt dann die 
    Darstellungs"anderungen, statt direkt durch Propagation des Controllers. 
    "Uber das ViewModel werden Komponenten der View 
    mit Daten des Model verbunden, dementsprechend muss das ViewModel auch 
    das Model modifizieren k"onnen und das Model das ViewModel "uber Ver"anderungen
    benachrichtigen.\\
"Ahnlich zum Model-View-Controller werden also Benutzer eingaben bis zum Model
    durch gereicht, jedoch Direkter als "uber den Controller, da Daten nahezu
    direkt in Verbindung stehen. Das Model f"ullt also inderekt die View durch
    die Zwischeninstanz eines ViewModel, das die Verbindungen herstellt.\\

\subsection{git}

Mit \# gekennzeichnete lines bezeichnen Kommentare.\\
Das Editieren der Datei, musste nicht angegeben werden, deshalb ist es 
    auskommentiert.\\

\subsubsection\ 
(i)\\
\begin{verbatim}
    # S1:
    $ git init
    # touch data.txt
    $ git add data.txt
    $ git commit -m "first commit"
    # S2:
    $ git checkout -b fork
    # S3:
    # echo f > data.txt
    $ git commit data.txt -m "fork commit"
    $ git checkout master
    # echo m > data.txt
    $ git commit data.txt -m "master commit"
    # S4:
    $ git merge fork
    # An dieser stelle werden wir auf einen merge-conflict aufmerksam gemacht
    # Konflikt loesen z.B. mit git mergetool
    $ git commit -m "resolved merge-conflict"
\end{verbatim}
\gap
(ii)\\
\begin{verbatim}
*   9e58136 (HEAD -> master) resolved merge-conflict
|\
| * bd51b50 (fork) fork commit
* | 34835a4 master commit
|/
* d1cdcc3 first commit
\end{verbatim}

\subsubsection\

(i)\\
\begin{verbatim}
    # S1:
    $ git init
    # touch data.txt
    $ git add data.txt
    $ git commit -m "first commit"
    # S2:
    $ git checkout -b fork
    # S3:
    # echo f > data.txt
    $ git commit data.txt -m "fork commit"
    $ git checkout master
    # echo m > data.txt
    $ git commit data.txt -m "master commit"
    # S4:
    $ git rebase master
    # An dieser Stelle werden wir auf den rebase-conflict aufmerksam gemacht
    # Konflikt loesen
    $ git add data.txt
    $ git rebase --continue
    # S5:
    $ git checkout master
    $ git merge fork
\end{verbatim}
\gap
(ii)\\
\begin{verbatim}
* c8ef802 (HEAD -> master, fork) fork commit
* 59b30b1 master commit
* 69762a1 first commit
\end{verbatim}

\subsubsection\

Eine Rebase-Operation erm"oglicht es Commits, die nicht zielf"uhrend oder
    sogar sch"adlich f"ur die Applikation/ das System waren aus der Historie
    zu l"oschen, damit also unzug"anglich zu machen und den Stand des Branches
    auf den der Produktionsreifen Version zu setzten.
    Der Commit-Verlauf ist also sauber und sogar frei von Merge-Commits.\\
Dies ist bei einer Merge-Operation nicht er Fall, jedoch muss man 
    hierbei die Branches nicht gleich zerst"oren und hat mehr Freiheiten bei der
    Weiterentwicklung. Man erh"alt durch das Aufgeben eines gewissen Gerades
    an "Ubersichtlichkeit ein stabileres Projekt, denn bei Rebase-Operationen
    kann schnell ein Fehler unterlaufen und eine Menge an Kontext verloren gehen.
    Ferner kann leicht durch eine Rebase-Operation der 
    Workflow ins stocken geraten.\\
Der Hauptunterschied ist also die "Ubersichtlichkeit des Graphen, die durch eine
    Rebase-Operation entsteht 
    und der h"ohere Informationsgehalt des Graphen, der durch die weiterhin als
    parallel identifizierbaren Commits bei einer Merge-Operation gewahrt wird.
\end{document}
