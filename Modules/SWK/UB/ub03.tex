\documentclass{article}
\textwidth=6in
\hoffset=0in
\voffset=0in

\usepackage{dcolumn}
\usepackage{afterpage}
\usepackage{pgf}
\usepackage{tikz}
\usepackage{pdflscape}
\usetikzlibrary{arrows,automata}
\usepackage[latin1]{inputenc}
\usepackage{ngerman}
\usepackage[a4paper, total={6in, 8in}]{geometry}
\usepackage{amsmath}
\usepackage{amssymb}
\usepackage{stmaryrd}
\usepackage{graphicx}
\usepackage{tikz}
\usetikzlibrary{automata, arrows, fit, calc}
\usepackage{pifont}
\usepackage{amssymb}
\usepackage{gensymb}
\usepackage[ampersand]{easylist}


\usepackage{listings}

\newcommand{\gap}{\ \\ \\}

\setcounter{section}{2}

% needs to be updated
\author{Max Springenberg, 177792}
\title{SWK\\
       "Ubungsblatt 3\\
       Gruppe 2: OH12 / 1.056}
\date{}
\setcounter{section}{3}


\begin{document}
\maketitle
\newpage

\subsection{Die Modellierungs-Hierarchie}
\subsubsection \
Ein Metamodell beschreibt Modelle.
    GraphML beschreibt xml-files, die als Graphen interpretiert werden k"onnen.
    Damit sind die Modelle von GraphML xml-files.\\
\subsubsection \ 
(i) die abstrakte Syntax von GraphML definiert die zugrundelegenden Datenobjekte.
    So behandelt GraphML Knoten, Kanten, bzw. Graphen im Allgemeinen, sowie
    ihre Darstellung.\\
(ii) Die konrete Syntax von GraphML definiert Schl"usserw"orter und die Grammatik
    der Sprache, z.B. die Schl"usselw"orter node, edge, graph, key, data und
    das Verwenden der xml-"ublichen Klammerung.\\
(iii) Die Semantik von GraphML ist durch das Auswerten der Inhalte f"ur die
    jeweiligen geklammerten Schl"usselw"orter zu Datenobjekten definiert.\\
\subsubsection \ 
XML ist Turingvollst"andig, damit k"onnen alle f"ur die 
    Informatik relevanten mathematischen Modelle beschrieben werden.\\
\subsubsection \ 
Wie gesagt ist XML Turingvollst"andig, damit kann man die Kontextsensitiven-Grammatikken
    als Metamodell f"ur Programmiersprachen im Allgemeinen erachten.\\
    XML ist in der Sprache der Kontextsitiven-Grammatikken enthalten,
    eine Modell-Instanz vom Metamodell `Kontextsensitive-Grammatik` ist 
    die Sprache XML.\\
Alternativ: XML Schema Definition Language, was im Internet
    als das ofizielle Metamodell von XML gelistet wird.\\

\newpage
\subsection{UML}
\newpage

\subsection{Anforderungs-DSLs}
\begin{verbatim}
(i)
requirement IMPLEMENTATIONTHREAD organization {
    implemented-in developers;
    urgency must;
    description "customers implementation-requirements";
    action follow;
};
(i.i)
requirement OPENSOURCE_PL organization {
    required-by IMPLEMENTATIONTHREAD;
    implemented-in developers;
    urgency must;
    description "open source programming language";
    action follow;
};
(i.ii)
requirement OPENSOURCE_ALL organization {
    required-by IMPLEMENTATIONTHREAD;
    implemented-in developers;
    urgency should;
    description "everything should be open-source";
    action follow;
};
(ii)
requirement DATAEXPORT product {
    implemented-in developers;
    urgency should;
    automatic;
    description "automatic data export";
    action export;
};
(iii)
// Der Einleitungstext konnte auch als Anforderung aufgefasst werden, 
// desshalb die zweiteilung der Anforderung.
requirement EXTENDTOOL system {
    implemented-in developers;
    urgency must;
    description "extend backup tool";
    action implement;
};
requirement PROHIBITDEL organization {
    required-by EXTENDTOOL;
    implemented-in developers;
    urgency must not;
    description "admin must not delete latest backup";
    action delete;
};
(iv)
requirement PROHIBITTOBI_WEB user {
    implemented-in component WEB_FRONTEND;
    urgency may not;
    usecase-for TOBI;
    description "don't let Tobi get access to salary-data";
    action access;
};
\end{verbatim}
\end{document}
