\documentclass{article}
\textwidth=6in
\hoffset=0in
\voffset=0in

\usepackage{dcolumn}
\usepackage{afterpage}
\usepackage{pgf}
\usepackage{tikz}
\usepackage{pdflscape}
\usetikzlibrary{arrows,automata}
\usepackage[latin1]{inputenc}
\usepackage{ngerman}
\usepackage[a4paper, total={6in, 8in}]{geometry}
\usepackage{amsmath}
\usepackage{amssymb}
\usepackage{stmaryrd}
\usepackage{graphicx}
\usepackage{tikz}
\usetikzlibrary{automata, arrows, fit, calc}
\usepackage{pifont}
\usepackage{amssymb}
\usepackage{gensymb}
\usepackage[ampersand]{easylist}


\usepackage{listings}

\newcommand{\gap}{\ \\ \\}

\setcounter{section}{2}

% needs to be updated
\author{Max Springenberg, 177792}
\title{SWK\\
       "Ubungsblatt 3\\
       Gruppe 2: OH12 / 1.056}
\date{}
\setcounter{section}{3}


\begin{document}
\subsection{Die Modellierungs-Hierarchie}
\subsubsection \
Ein Metamodell beschreibt Modelle.
    GraphML beschreibt xml-files, die als Graphen interpretiert werden k"onnen.
    Damit sind die Modelle von GraphML xml-files.\\
\subsubsection \ 
(i) die abstrakte Syntax con GraphML definiert die zugrundelegenden Datenobjekte.
    So behandelt GraphML Knoten, Kanten, bzw. Graphen im allgemeinen, sowie
    ihre Darstellung.\\
(ii) Die Konrete Syntax von GraphML definiert Schl"usserw"orter und die Grammatik
    der Sprache, z.B. die Schl"usselw"orter node, edge, graph, key, data und
    das Verwenden der xml-"ublichen Klammerung.\\
(iii) Die Semantik von GraphML ist durch das Auswerten der geklammerten 
    Schl"usselw"orter zu Datenobjekten definiert.
    So ist durch die Grammatik der konkreten Syntax gegeben, dass die dieser
    zugrundeliegende abstrakte Syntax als die entsprechenden Datenobjekte
    interpretiert werden kann.\\
\subsubsection \ 
XML ist Turingvollst"andig, damit k"onnen alle f"ur die 
    Informatik relevanten mathematischen Modelle beschrieben werden, 
    auch wenn wir mit den Schl"usselw"ortern von GraphML auskommen m"ussten,
    solange wir die abstrakte Syntax "andern.\\
\subsubsection \ 
Wie gesagt ist XML Turingvollst"andig, damit kann man die Kontextsensitiven-Grammatikken
    als Metamodell f"ur Programmiersprachen im Allgemeinen erachten.\\
    XML ist in der Sprache der Kontextsitiven-Grammatikken enthalten,
    eine Modell-Instanz vom Metamodell `Kontextsensitive-Grammatik` ist 
    die Sprache XML.\\
Alternativ: XML Schema Definition Language.\\

\newpage
\subsection{UML}
\newpage

\subsection{Anforderungs-DSLs}
\begin{verbatim}
(i.i)
requirement OPENSOURCE_PL organization {
    implemented-in developers;
    urgency must;
    description "open source programming language";
    action follow;
};
(i.ii)
requirement OPENSOURCE_ALL organization {
    implemented-in developers;
    urgency should;
    description "everything should be open-source";
    action follow;
};
(ii)
requirement DATAEXPORT system {
    implemented-in developers;
    urgency should;
    automatic;
    description "automatic data export";
    action export;
};
(iii)
requirement EXTENDTOOL product {
    implemented-in developers;
    urgency must;
    description "extend backup tool";
    action implement;
};
requirement PROHIBITDEL organization {
    required-by EXTENDTOOL;
    implemented-in global;
    urgency must not;
    description "admin must not delete latest backup";
    action delete;
};
(iv)
requirement PROHIBITTOBI_WEB organization {
    implemented-in component WEB_FRONTEND;
    urgency may;
    usecase-for TOBI;
    description "don't let tobi get access to salary-data";
    action deny;
};
\end{verbatim}
\end{document}
