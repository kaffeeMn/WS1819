\documentclass{article}
\textwidth=6in
\hoffset=0in
\voffset=0in

\usepackage{dcolumn}
\usepackage{afterpage}
\usepackage{pgf}
\usepackage{tikz}
\usepackage{pdflscape}
\usetikzlibrary{arrows,automata}
\usepackage[latin1]{inputenc}
\usepackage{ngerman}
\usepackage[a4paper, total={6in, 8in}]{geometry}
\usepackage{amsmath}
\usepackage{amssymb}
\usepackage{stmaryrd}
\usepackage{graphicx}
\usepackage{tikz}
\usetikzlibrary{automata, arrows, fit, calc}
\usepackage{pifont}
\usepackage{amssymb}
\usepackage{gensymb}
\usepackage[ampersand]{easylist}


\newcommand{\gap}{\ \\ \\}
\newcommand{\chart}[6]{
    \node[round, 
          minimum width=#5, 
          minimum height=#6] (#1) at (#3,#4) {chart};
    \node[rect] (#2) at (#3,#4) {title};
}
\newcommand{\sepline}[2]{
}

% needs to be updated
\author{Max Springenberg, 177792\\
        Daniel Sonnabend, 190748}
\title{SWK\\
       "Ubungsblatt 1}

\begin{document}
\maketitle
\newpage

\subsection{Anforderungen}

\subsubsection\
Klassifizierung der Anforderungen:\\
\\
\begin{tabular}{|l|l|l|l|l|}
    \hline
    \multicolumn{2}{|c|}{Funktional} & \multicolumn{3}{c|}{nicht Funktional}\\
    \hline
    Benuztzer       & System  & Unternehmen & Produkt       & Extern\\
    \hline
    (i), (v), (vii) & (iv)    & (iii)       & (vi), (vii)   & (ii)\\
    \hline
\end{tabular}

\subsubsection\

(i) Welche Aktionen muss ein Benutzer ausf"uhren, um sich zu registrieren?\\
(ii) In welchem Format werden Benutzerdaten gespeichert?\\
(iii) M"ussen Passw"orter besonders gesch"utzt werden?\\
(iv) Wie wird sichergestellt, dass die Entwickler des Software keinen 
     Zugriff auf Passw"orter erhalten?\\
\gap
Funktionale Benutzeranforderung:\\
(i) Der Benutzer muss einen einzigartigen Benutzernamen ausw"ahlen k"onnen.\\
\\
Funktionale Systemanforderung:\\
(ii) Das System sollte f"ahig sein Nutzerdaten in einem csv-Format zu speichern.\\
\\
Nicht funktional externe Anforderung:\\
(iii) Beim Speichern von Passw"ortern und weiteren sensiblen Daten m"ussen
      Anforderungen des deutschen und europ"aischen Rechts gewahrt werden.\\
\\
Nicht funktionale Unternehmensanforderung:\\
(iv) Schl"ussel der Cypher-Texte m"ussen f"ur Entwickler uneinsehbar bleiben.\\

\subsection{}

\subsection{2- und 3-Punktsch"atzung}
Es wurde in der Vorlesung definiert:\\
$E_{2-Punkt}(x) = \frac{a+b}{2}, S(x) = \frac{b-a}{6}$\\
$E_{3-Punkt}(x) = \frac{a+4*c+b}{2}, S(x) = \frac{b-a}{6}$\\
$E = \Sigma_i E_i$\\
\gap
Durch einzetzen folgt f"ur:\\
(i)/(ii)\\
\begin{flalign*}
    S   &= \sqrt{(\frac{60-10}{6})^2 + (\frac{160-30}{6})^2 
            + (\frac{12-4}{6})^2 + (\frac{20-8}{6})^2}&&\\
        &= \sqrt{8,3^2 + 21.7^2 + 1.3^2 + 2^2} = 23.4&&\\
\end{flalign*}
(i)\\
\begin{flalign*}
    E &= \frac{10+60}{2} + \frac{30+160}{2} + \frac{4+12}{2} + \frac{8+20}{2}&&\\
      &= 35 + 95 + 8 + 14 = 152&&\\
\end{flalign*}
(ii)\\
\begin{flalign*}
    E &= \frac{10+4*40+60}{2} + \frac{30+4*80+160}{2} 
       + \frac{4 + 4*8 + 12}{2} + \frac{8 + 4*10 + 20}{2}&&\\
      &= 115 + 255 + 26 + 34 = 430&&\\
\end{flalign*}
\end{document}
