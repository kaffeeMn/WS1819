\documentclass{article}
\textwidth=6in
\hoffset=0in
\voffset=0in

\usepackage{dcolumn}
\usepackage{afterpage}
\usepackage{pgf}
\usepackage{tikz}
\usepackage{pdflscape}
\usetikzlibrary{arrows,automata}
\usepackage[latin1]{inputenc}
\usepackage{ngerman}
\usepackage[a4paper, total={6in, 8in}]{geometry}
\usepackage{amsmath}
\usepackage{amssymb}
\usepackage{stmaryrd}
\usepackage{graphicx}
\usepackage{tikz}
\usetikzlibrary{automata, arrows, fit, calc}
\usepackage{pifont}
\usepackage{amssymb}
\usepackage{gensymb}
\usepackage[ampersand]{easylist}


\newcommand{\gap}{\ \\ \\}

% node of size 1x1 with 
% #1 : ID
% #2 : min
% #3 : max
% #4 : buffer
% #5, #6 : coords of center
\newcommand{\cpnd}[6]{
    \node[] ()  at (#5,    #6+0.25) {#1};
    \node[] ()  at (#5-0.25,#6)     {#2};
    \node[] ()  at (#5+0.25,#6)     {#3};
    \node[] ()  at (#5,    #6-0.25) {#4};
    %\node[rounded corners=1.5mm, minimum width=3cm, draw, anchor=center,
    %      minimum width=1cm, 
    %      minimum height=1cm] (#1) at (#5,#6) {};
}

% needs to be updated
\author{Max Springenberg, 177792}
\title{SWK\\
       "Ubungsblatt 1}

\begin{document}
\maketitle
\newpage

\subsection{Anforderungen}

\subsubsection\
Klassifizierung der Anforderungen:\\
\\
\begin{tabular}{|l|l|l|l|l|}
    \hline
    \multicolumn{2}{|c|}{Funktional} & \multicolumn{3}{c|}{nicht Funktional}\\
    \hline
    Benuztzer       & System  & Unternehmen & Produkt       & Extern\\
    \hline
    (i), (v), (vii),(viii) & (iv)    & (iii)       & (vi)             & (ii)\\
    \hline
\end{tabular}

\subsubsection\

(i) Welche Aktionen muss ein Benutzer ausf"uhren, um sich zu registrieren?\\
(ii) In welchem Format werden Benutzerdaten gespeichert?\\
(iii) M"ussen Passw"orter besonders gesch"utzt werden?\\
(iv) Wie wird sichergestellt, dass die Entwickler des Software keinen 
     Zugriff auf Passw"orter erhalten?\\
\gap
Funktionale Benutzeranforderung:\\
(i) Der Benutzer muss einen einzigartigen Benutzernamen ausw"ahlen k"onnen.\\
\\
Funktionale Systemanforderung:\\
(ii) Das System sollte f"ahig sein Nutzerdaten in einem csv-Format zu speichern.\\
\\
Nicht funktional externe Anforderung:\\
(iii) Beim Speichern von Passw"ortern und weiteren sensiblen Daten m"ussen
      Anforderungen des deutschen und europ"aischen Rechts gewahrt werden.\\
\\
Nicht funktionale Unternehmensanforderung:\\
(iv) Schl"ussel der Cypher-Texte m"ussen f"ur Entwickler uneinsehbar bleiben.\\


\subsection{}
siehe Anhang

\subsection{2- und 3-Punktsch"atzung}
\subsubsection\
Es wurde in der Vorlesung definiert:\\
$E_{2-Punkt}(x) = \frac{a+b}{2}, S(x) = \frac{b-a}{6}$\\
$E_{3-Punkt}(x) = \frac{a+4*c+b}{2}, S(x) = \frac{b-a}{6}$\\
%$E = \Sigma_i E_i$\\
\\
Da die Komponenten unabh"angig voneinander arbeiten, reicht es aus die maximale
    Komponente `Gesch"aftslogik` betrachten und es m"ussen keine Summen 
    gebildet werden.\\
\newpage
\  \\
Durch einsetzen folgt f"ur:\\
(i)/(ii)\\
$S = \frac{160-30}{6} = 31,7$\\
%\begin{flalign*}
%    S   &= \sqrt{(\frac{60-10}{6})^2 + (\frac{160-30}{6})^2 
%            + (\frac{12-4}{6})^2 + (\frac{20-8}{6})^2}&&\\
%        &= \sqrt{8,3^2 + 31.7^2 + 1.3^2 + 2^2} = 23.4&&\\
%\end{flalign*}
(i)\\
$E = \frac{30 + 160}{2} = 95$\\
%\begin{flalign*}
%    E &= \frac{10+60}{2} + \frac{30+160}{2} + \frac{4+12}{2} + \frac{8+20}{2}&&\\
%      &= 35 + 95 + 8 + 14 = 152&&\\
%\end{flalign*}
(ii)\\
$E = \frac{30 + 4*80 + 160}{2} = 255$\\
%\begin{flalign*}
%    E &= \frac{10+4*40+60}{2} + \frac{30+4*80+160}{2} 
%       + \frac{4 + 4*8 + 12}{2} + \frac{8 + 4*10 + 20}{2}&&\\
%      &= 115 + 255 + 26 + 34 = 430&&\\
%\end{flalign*}

\subsubsection\
Vorteil einer 3-Punktsch"atzung ist deren Genauigkeit, bei guter Einsch"atzung
    des m"oglichen Aufwands. Jedoch ist die 3-Punktsch"atzung vergleichsweise
    schlecht wenn man nicht absehen kann wie gro"s der Aufwand ist.\\
Vorteil einer 2-Punktsch"atzung ist deren nicht all zu schlechte Genauigkeit, 
    bei schlechter Einsch"atzung des m"oglichen Aufwands. Jedoch ist deren
    Genauigkeit nicht so gut wie die der 3-Punktsch"atzung bei gut absehbarem
    Aufwand.\\
\end{document}
