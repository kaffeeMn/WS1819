\section{Teil 2 - Spezifikations- und Modellierungssprachen}

\subsection{Von Neumann Modell (Deadlocks)}

\begin{enumerate}
    \item mutual exclusion
    \item no preemption
    \item holding recources
    \item circular waiting
\end{enumerate}



\subsection{Undefined Components}

\subsubsection{Anforderungen an Spezifikations- und Modellierungssprachen/techniken}

Meschen sind nicht darauf ausgelegt Systeme zu verstehen, die mehr als 5 Komplexe Objekte enthalten.
Die Meisten Systeme fordern jedoch mehr.
Hilfe bietet eine Hierarchy für die Spraceh/ Technik.
\begin{itemize}
    \item Behavioral: states, processes, procedures
    \item Structural: processors, racks, printed circuit boards
    \item Component-based design: Das System muss von Komponenten designed sein,
            die Synchronisiert argieren können.
    \item Timing: Erforderliche Spezifikationen \begin{itemize}
        \item Measured elapsed Time: Check nach vergangener Zeit
        \item Means for delaying Process: Prozesse können schlafen gelegt werden
        \item Possibility to specify timeouts: In einer spezifizierten maximalen Laufzeit for Timeout bleiben
        \item Methods for specifying deadlines: Programme sollen vor einer Angegebenen Zeit terminieren
    \end{itemize}
    \item support for design: kann unterteilt werden in \begin{itemize}
        \item State-oriented behaviour: Verhalten, wie das von Automaten (States)
        \item Event-handling: externe oder interne Events lösen Berechnungen aus
        \item Exception-oriented behaviour: Imgang mit Fehlern
    \end{itemize}
\end{itemize}



\subsection{SUD}

Das system under design (SUD) wird oftmals wie auch in der Software-Konstruktion üblich als Anfragetext
vorgelegt.
Dabei sind Anforderungen and das SUD:
\begin{itemize}
    \item Machine-readable
    \item Version-management
    \item Dependency analysis
\end{itemize}
Applikations-Interaktionen können mittels UML modelliert werden. (Sequenz-/Use-Case Diagramme)


\subsection{Communicating finite state machines}

\subsubsection{State Charts}
State Charts bieten eine vereinfachte Schreibweise komplexer Automaten.
\\ \\
In einem State Chart wir unteschieden zwischen super- und supstates.
Substates werden in einem superstate durchlaufen, der superstate selbst betritt entweder immer seinen
ersten substate oder den letzten eingetretenen substate, je nachdem, ob mit einer History gearbeitet wird
oder nicht.
\\ \\
Superstates heißen OR-superstate wenn genau ein substate aktiv ist und AND-superstate, wenn substates conkurrierend 
aktuv sind.
\\ \\
\\
Kanten werden mit der Konvetion 'event[condition]/reaction' (z.B. 'service-off[not in lProc]/service:=0') beschriftet.
Die Evaluierung dieser Labels erfolgt mithilfe der State-Mate Semantik.
In dieser existieren 3 Phasen.
\begin{enumerate}
    \item Effekte externer Veränderungne und Konditionen evaluieren
    \item Die Menge an Traisitionen während des aktuellen Schrittes und rechte Seiten (Zuweisungen) werden berechnet
    \item Transitionen werden aktiv, die Variablen bekommen ihre neuen Werte zugewiesen
\end{enumerate}
Hierbei erhalten wieder ein bestimmtes Verfahren durch die Trennung der Phasen 2 und 3.


\subsubsection{SDL}

SDL wird benutzt um asynchrone Nachrichtenvermittlung (asynchronous message passing communication) zu modellieren
\\ \\
Kommunikation zwischen SDLs basiert auf die Nachrichtenverteilung von Signalen(=input+output) mittels einer FIFO-queue.
Wenn Signale gleichzeitig ankommen ist deren Reihenfolge in der FIFO-queue nicht deterministisch/ unbekannt.



\subsection{Datenfluss}

Das Modellieren eines Datenflusses umfasst wie Daten in einem Informationssystem bewegt werden.
Dazu werden unter anderem:
\begin{itemize}
    \item Przesse (Aktivitäten, die Daten transformieren)
    \item Datenbanken/ data stores
    \item Externe Entitäten (was/ wer sendet/ empfängt daten ins/vom System)
    \item Datenflusse (Routen des Datenflusses)
\end{itemize}
betrachtet.


\subsubsection{Kahn process networks (KPN)}

Kahn Netzwerke haben jeweils mindestens einen Input- und Output-Kanal.
Einschränkungen:
\begin{itemize}
    \item Es kann nicht für verfügbare Daten for einer versuchten Leseoperation gecheckt werden
    \item Ein Prozess kann immer nur die Ports nacheinander lesen
\end{itemize}
Daraus folgt:
Die Reihenfolge der Leseoperationen erfolgt nicht über die Ankunft von Daten. Also sind KPNs deterministisch.
Ferner sind KPNs Turingvollständig, aber nicht Deadlock-frei.

\subsubsection{Synchronous data flow (SDF)}

Weniger stark als KPN, aber besser zu analysieren sind SDFs.
SDFs besieren auf einer globalen Uhr, diese entscheidet wann Token `gefeuert` werden.
In einem SDF Graphen/Netzwerk werden die knoten als Akteure bezeichnet.
Ein Akteur ist bereit, wenn genügend Tokens für ihn vorhanden sind.
Ein akteur verbrauch und produziert Token (Input/ Output).
\\ \\
Gibt es Deadlocks in SDFs?\\
Nein SDFs sind Deadlock-frei.
Dafür sind SDFs aber nicht Turingvollständig.
