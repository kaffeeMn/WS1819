\section{Teil 2 - Spezifikations- und Modellierungssprachen}

\subsection{Undefined Components}

\subsubsection{Anforderungen an Spezifikations- und Modellierungssprachen/techniken}

Meschen sind nicht darauf ausgelegt Systeme zu verstehen, die mehr als 5 Komplexe Objekte enthalten.
Die Meisten Systeme fordern jedoch mehr.
Hilfe bietet eine Hierarchy für die Spraceh/ Technik.
\begin{itemize}
    \item Behavioral: states, processes, procedures
    \item Structural: processors, racks, printed circuit boards
    \item Component-based design: Das System muss von Komponenten designed sein,
            die Synchronisiert argieren können.
    \item Timing: Erforderliche Spezifikationen \begin{itemize}
        \item Measured elapsed Time: Check nach vergangener Zeit
        \item Means for delaying Process: Prozesse können schlafen gelegt werden
        \item Possibility to specify timeouts: In einer spezifizierten maximalen Laufzeit for Timeout bleiben
        \item Methods for specifying deadlines: Programme sollen vor einer Angegebenen Zeit terminieren
    \end{itemize}
    \item support for design: kann unterteilt werden in \begin{itemize}
        \item State-oriented behaviour: Verhalten, wie das von Automaten (States)
        \item Event-handling: externe oder interne Events lösen Berechnungen aus
        \item Exception-oriented behaviour: Imgang mit Fehlern
    \end{itemize}
\end{itemize}


\subsection{SUD}

Das system under design (SUD) wird oftmals wie auch in der Software-Konstruktion üblich als Anfragetext
vorgelegt.
Dabei sind Anforderungen and das SUD:
\begin{itemize}
    \item Machine-readable
    \item Version-management
    \item Dependency analysis
\end{itemize}
Applikations-Interaktionen können mittels UML modelliert werden. (Sequenz-/Use-Case Diagramme)



\subsection{Communicating finite state machines}

\subsubsection{State Charts}
State Charts bieten eine vereinfachte Schreibweise komplexer Automaten.
\\ \\
In einem State Chart wir unteschieden zwischen super- und supstates.
Substates werden in einem superstate durchlaufen, der superstate selbst betritt entweder immer seinen
ersten substate oder den letzten eingetretenen substate, je nachdem, ob mit einer History gearbeitet wird
oder nicht.



\subsubsection{SDL}
