\section{Teil 1 - ES CPS}

\subsection{ES}

Ein eingebettetes System (ES) ist ein System, das Informationen verabeitet
und in ein größeres Produkt eingebettet ist.
Ein ES stellt also Software bereit, die durch die Verarbeitung von (Sensor-)
Signalen/ Informationen ein Produkt im Features, bzw. Verbesserungen ergänzt.


\subsection{CPS}

Cyber-Physical-Systems (CPS) sind im Wesentlichen Systeme, die zuzüklich 
auf Kommunikation ausgelegt sind. (smart-home, smart-$\ldots$)


\subsection{Charakteristika}

\subsubsection{Dependability}
CPS und ES müssen verlässlich sein.
Deshalb definieren wir folgende Güten:
\begin{flalign*}
    \text{Reliability } R(t)        & = \text{Wahrscheinlichkeit, dass das System zu} t=0 \text{funktioniert}\\
    \text{Maintainability } M(d)    & = \text{Wahrscheinlichkeit, dass das System} d \text{Zeiteinheiten nach einem Fehler funktioniert}\\
    \text{Availability } A(t)       & = \text{Wahrscheinlichkeit, dass das System zur Zeit } t \text{funktioniert}\\
    \text{Safety }                  & = \text{Das System verursacht keinen Schaden}\\
    \text{Security }                & = \text{Das System bedient sich authentischer und vertraulicher Kommunikation}\\
\end{flalign*}


\subsubsection{Effizienz}

Energieeffizienz:\\
Für ES/ CPS ist die Energieeffizienz von bedeutung, die sie oftmals auf Batteriebetrieb angewiesen sind.
Energieeffiziente Prozessoren verlangen jedoch eine aufwendigere Implementierung des ES/ CPS.
Hierbei gilt es für Produzenten/ Firmen einen Mittelweg zu finden.
\\ \\
real-time constraints:\\
Wird bei z.B. einem Auto ein real-time constraint nicht eingehalten können die Folgen verherend sein.
Dies wäre auch ein Beispiel für einen harten real-time constraint.
Alle nicht in Katastrophen resultierende real-time constraints sind soft.
\\ \\
Weite Faktoren hinsichtlich der Effizienz sind: Gewicht der Hardware, Kosten der Hardware und Entwicklung, Code-Größe/ Länge

\subsection{Anwendung}

ES/ CPS finden Anwendung in:\\
\begin{enumerate}
    \item Transport:\\
          Die Automobilindustrie verwendet ES mit ABS, ESP, etc., die 
          Flugzeugindustrie benutzt ES für Flugkontrolle, Kollisionsvermeidung,
          Autopiloten, etc.. Auch Schiffe und Z"uge verwenden ES für 
          Sicherheits- und Kontroll/ Navigations Features.\\
    \item Logistik:\\
          Die Logistik verwendet ES für Radio-Frequency-Identification,
          mobile Kommunikation, etc..\\
    \item Fabriken:\\
          Fabriken verwenden ES in 'social machines`, Maschinen de sich selbst 
          konfigurieren und/ oder distributieren.\\
    \item Structural Safety:\\
          Darunter fällt Regulation des Wasserstandes eines Damms,
          Überwachung von Brücken/ Vulkanen, sowie die
          Neigung von Hochhausern bei Erdbeben.\\
    \item Smart Home:\\
          Häuser können ES für zero energy buildings, safety/ security, 
          comfort, ambient assited living (selbst regulierende Fenster etc.)
          benutzen.\\
\end{enumerate}
