\documentclass{article}
\textwidth=6in
\hoffset=0in
\voffset=0in

\usepackage{afterpage}
\usepackage{pgf}
\usepackage{tikz}
\usepackage{pdflscape}
\usetikzlibrary{arrows,automata}
\usepackage[latin1]{inputenc}
\usepackage{ngerman}
\usepackage[a4paper, total={6in, 8in}]{geometry}
\usepackage{amsmath}
\usepackage{amssymb}
\usepackage{stmaryrd}
\usepackage{graphicx}
\usepackage{tikz}
\usetikzlibrary{automata, arrows, fit, calc}
\usepackage{pifont}
\usepackage{amssymb}
\usepackage{gensymb}
\usepackage[ampersand]{easylist}


\newcommand{\gap}{\ \\ \\}
\newcommand{\chart}[6]{
    \node[round, 
          minimum width=#5, 
          minimum height=#6] (#1) at (#3,#4) {chart};
    \node[rect] (#2) at (#3,#4) {title};
}
\newcommand{\sepline}[2]{
}

% needs to be updated
\author{Max Springenberg, 177792\\
        Daniel Sonnabend, 190748}
\title{\
    ES "Ubungsblatt 3\\
    Gruppe Fr. 8-10
    }
\setcounter{section}{3}
\date{}

\begin{document}

\maketitle
\newpage

\subsection{Fragen}

Wie wird ein Task im OSEK-Betriebssystem terminiert?\\
Ein Task terminiert genau dann, wenn er sich selbst terminiert. Es gibt keine
    Zwischentransition zum Zustand wait.\\
\gap
OSEK unterscheidet zwischen zwei Tasktypen. Nennen Sie diese und erl"autern Sie 
    den Unterschied.\\
\\
Typ 1: Extended Task:\\
Der Extended Task hat die vier States running, ready, waiting und suspended.
    Im State runnuing wird die CPU zugeteilt, nur ein Prozess kann in diesem
    State sein. Bei ready wird gewartet, bis das Scheduling-Verfahren den Prozess
    in running einteilt. Bei waiting wartet der Task auf mindestens ein Event 
    und bei suspended wird der Prozess passiv, kann aber wieder aktiviert werden.\\
\\
Typ 2: Basic Task\\
Die Funktionalit"at eines Basic Task ist "ahnlich zu der des Extended Task.
    Jedoch existiert kein waiting State. Damit fallen der waiting State und 
    alle zu/ von diesem ein- und ausgehenden Transitionen weg.
\gap
In welchem Zustand befindet sich ein vom Scheduler aktivierter Task? Was 
    zeichnet diesen zudem aus?\\
Ein vom Scheduler aktivierter Prozess befindet sich im State running, dieser
    ist dadurch ausgezeichnet, dass immer nur ein Prozess in diesem State sein
    kann.\\
\end{document}
