\documentclass{article}
\textwidth=6in
\hoffset=0in
\voffset=0in

\usepackage{afterpage}
\usepackage{pgf}
\usepackage{tikz}
\usepackage{pgfplots}
\usepackage{pdflscape}
\usetikzlibrary{arrows,automata}
\usepackage[utf8x]{inputenc}
\usepackage[a4paper, left=2cm, right=3cm, top=2cm, bottom=2cm]{geometry}
\usepackage{amsmath}
\usepackage{amssymb}
\usepackage{stmaryrd}
\usepackage{graphicx}
\usepackage{tikz}
\usetikzlibrary{automata, arrows, fit, calc}
\usepackage{pifont}
\usepackage{amssymb}
\usepackage{gensymb}
\usepackage[ampersand]{easylist}


\newcommand{\gap}{\ \\ \\}
\newcommand{\chart}[6]{
    \node[round, 
          minimum width=#5, 
          minimum height=#6] (#1) at (#3,#4) {chart};
    \node[rect] (#2) at (#3,#4) {title};
}
\newcommand{\sepline}[2]{
}

% needs to be updated
\author{Max Springenberg, 177792\\
        Daniel Sonnabend, 190748}
\title{\
    ES Übungsblatt 12\\
    Gruppe Fr. 8-10
    }
\setcounter{section}{12}
\date{}

\begin{document}

\maketitle

% A1
\subsection\
$\alpha^l, \alpha^u$ sind wie folgt definiert.\\
\begin{flalign*}
	\alpha^l(\Delta) & \overset{def}{=} \underset{\lambda \geq 0, \forall R}{inf} \{R(\Delta + \lambda) - R(\lambda)\}&&\\
	\alpha^u(\Delta) & \overset{def}{=} \underset{\lambda \geq 0, \forall R}{sup} \{R(\Delta + \lambda) - R(\lambda)\}&&\\
\end{flalign*}

%A2
\subsection\
$\beta^l, \beta^u$ sind wie folgt definiert.
\begin{flalign*}
	\beta^l(\Delta) & \overset{def}{=} \underset{\lambda \geq 0, \forall C}{inf} \{C(\Delta + \lambda) - C(\lambda)\}&&\\
	\beta^u(\Delta) & \overset{def}{=} \underset{\lambda \geq 0, \forall C}{sup} \{C(\Delta + \lambda) - C(\lambda)\}&&\\
\end{flalign*}

% A3
\subsection\
\begin{tikzpicture}
	\begin{axis}[xlabel=$\Delta$, ylabel=$\alpha^u(\Delta)$
	                    , xtick={0,0.4,1,1.4,2,2.4,3}, xticklabels={$0$, $d$, $p$, $p+d$, $2p$, $2p+d$, $3p$}
	                    , axis lines = middle, xmin=0, ymin=0]
	\addplot[color=blue] coordinates{
		(0,1)
		(0.4, 1)
		(0.4, 2)
		(1,2)
		(1,3)
		(1.4, 3)
		(1.4, 4)
		(2,4)
		(2,5)
		(2.4,5)
		(2.4, 6)
		(3,6)
		(3,7)
		(3.4,7)
	};
	\draw[red, dashed, thick](axis cs:0.4,0) -- (axis cs:0.4,1);
	\draw[red, dashed, thick](axis cs:1.4,0) -- (axis cs:1.4,3);
	\draw[red, dashed, thick](axis cs:2.4,0) -- (axis cs:2.4,5);
	\end{axis}
\end{tikzpicture}

\subsection\
Konkret können wir $\beta^u, \beta^l$ wie folgt interpretieren:\\
\begin{flalign*}
    \beta^u(\Delta) & B \{\lceil\frac{\Delta}{\overset{-}{c}}\rceil s_i, \Delta - \lfloor\frac{\Delta}{\overset{-}{c}}\rfloor (\overset{-}{c} - s_i)\}&&\\
    \beta^l(\Delta) & B \{\lfloor\frac{\Delta}{\overset{-}{c}}\rfloor s_i, \Delta - \lfloor\frac{\Delta}{\overset{-}{c}}\rfloor (\overset{-}{c} - s_i)\}&&
\end{flalign*}
Daraus resultiert das folgende Diagramm für das Beispiel:\\
\begin{tikzpicture}
	\begin{axis}[xlabel=$\Delta$, ylabel=service
	                    , xtick={1, 2, 3}, xticklabels={$s$, $p$, $s+p$}
	                    , ytick={1, 2, 3}, yticklabels={$b \cdot s$, $2b \cdot s$,  $3b \cdot s$}
                        , axis lines = middle, xmin=0, ymin=0]
	\addplot[color=blue] coordinates{
        (0,0)
        (1,1)
        (3,1)
        (4,2)
        (7,2)
        (8,3)
        (10,3)
	};
	\addplot[color=green] coordinates{
        (0,0)
        (2,0)
        (3,1)
        (5,1)
        (7,2)
        (10,2)
	};
	\legend{$\beta^u$,$\beta^l$}
	\end{axis}
\end{tikzpicture}

\subsection\
\begin{tabular}{l|l}
    partitioniert                               &                       global\\ \hline
    Jeder Task hat einen Prozessor              & Jobs können auf beliebigen Prozessoren laufen\\
    Prozessor schduled eigene Prozesse          & Es gibt eine globale "ready queue"\\
    kein on-line Overhead                       & $M$ höchst priorisierten Jobs werde ausgeführt, mit $M \overset{def}{=} \#$ Prozessoren\\
                                                & hoher on-line Overhead\\
\end{tabular}
\end{document}
