\documentclass{article}
\textwidth=6in
\hoffset=0in
\voffset=0in

\usepackage{afterpage}
\usepackage{pgf}
\usepackage{tikz}
\usepackage{pdflscape}
\usetikzlibrary{arrows,automata}
\usepackage[latin1]{inputenc}
\usepackage{ngerman}
\usepackage[a4paper, total={6in, 8in}]{geometry}
\usepackage{amsmath}
\usepackage{amssymb}
\usepackage{stmaryrd}
\usepackage{graphicx}
\usepackage{tikz}
\usetikzlibrary{automata, arrows, fit, calc}
\usepackage{pifont}
\usepackage{amssymb}
\usepackage{gensymb}
\usepackage[ampersand]{easylist}


\newcommand{\gap}{\ \\ \\}
\newcommand{\chart}[6]{
    \node[round, 
          minimum width=#5, 
          minimum height=#6] (#1) at (#3,#4) {chart};
    \node[rect] (#2) at (#3,#4) {title};
}
\newcommand{\sepline}[2]{
}

% needs to be updated
\author{Max Springenberg, 177792\\
        Daniel Sonnabend, 190748}
\title{\
    ES "Ubungsblatt 5\\
    Gruppe Fr. 8-10
    }
\setcounter{section}{5}
\date{}

\begin{document}

\maketitle
\newpage

\subsection{Vorbereitung}
a.)\\
Es wurde von dem \glqq goal of portable software\grqq, 
    also das die Software nicht an eine 
    bestimmte Hardware gebunden, sondern portierbar ist, gesprochen.\\
\\
b.)\\
Der Datentyp ist UINT32, also ein 32 Bit Integer.
    Dementsprechend ist auch der Wert der niedrigsten Priorit"at $0$.\\
\\
c.)\\
In einem Task k"onnen RESOURCE, EVENT , MESSAGE mehrmals definiert werden.\\
\end{document}
