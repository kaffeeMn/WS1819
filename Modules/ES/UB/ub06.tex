\documentclass{article}
\textwidth=6in
\hoffset=0in
\voffset=0in

\usepackage{afterpage}
\usepackage{pgf}
\usepackage{tikz}
\usepackage{pdflscape}
\usetikzlibrary{arrows,automata}
\usepackage[latin1]{inputenc}
\usepackage{ngerman}
\usepackage[a4paper, total={6in, 8in}]{geometry}
\usepackage{amsmath}
\usepackage{amssymb}
\usepackage{stmaryrd}
\usepackage{graphicx}
\usepackage{tikz}
\usetikzlibrary{automata, arrows, fit, calc}
\usepackage{pifont}
\usepackage{amssymb}
\usepackage{gensymb}
\usepackage[ampersand]{easylist}


\newcommand{\gap}{\ \\ \\}
\newcommand{\chart}[6]{
    \node[round, 
          minimum width=#5, 
          minimum height=#6] (#1) at (#3,#4) {chart};
    \node[rect] (#2) at (#3,#4) {title};
}
\newcommand{\sepline}[2]{
}

% needs to be updated
\author{Max Springenberg, 177792\\
        Daniel Sonnabend, 190748}
\title{\
    ES "Ubungsblatt 6\\
    Gruppe Fr. 8-10
    }
\setcounter{section}{6}
\date{}

\begin{document}

\maketitle
\\
\gap
\\
\subsection{Vorbereitung}
a) Damit ein Alarm zu Systemstart erzeugt wird muss einen Counter und einen Task
    bekommen.\\
\gap
b) ALARMTIME ist die Zeit, zu der der Alarm das erste mal abl"auft.
    CYCLETIME ist die Zeit eines  Zyklus in einem zyklischen Alarm.\\
\gap
c) Der Counter misst in `ticks`.
    Damit gibt der Counter einen ganzzahligen Wert von vergangenen `ticks` an.\\
\end{document}
