\documentclass{article}
\textwidth=6in
\hoffset=0in
\voffset=0in

\usepackage{ngerman}
\usepackage[a4paper, total={6in, 8in}]{geometry}
\usepackage{amsmath}
\usepackage{amssymb}
\usepackage{stmaryrd}
\usepackage{graphicx}
\usepackage{tikz}
\usetikzlibrary{automata, arrows}
\usepackage{pifont}
\usepackage{amssymb}
\usepackage{gensymb}
\usepackage[ampersand]{easylist}

% needs to be updated
\author{Max Springenberg, 177792}
\title{\
    ES "Ubungsblatt 1
    }
\setcounter{section}{1}
\date{}

\begin{document}

\maketitle
\newpage

\subsection{Was ist ein eingebettetes System? Bitte geben Sie keine Definition 
    an, sondern beantworten Sie diese Frage in Ihren eigenen Worten.}

Ein eingebettetes System ist ein System, das Informationen verarbeitet und in 
    ein gr"o"seres Produkt eingebettet ist.\\
Ein eingebettetes System stellt also Software bereit, die durch die Verarbeitung
    von (Sensor-)Signalen/ Informationen ein Produkt um Features, bzw. 
    Verbesserungen erg"anzt.\\

\subsection{Nennen Sie mindestens zwei Anwendungsgebiete eingebetteter Systeme}

Eingebette Systeme(ES) finden Anwendung in:\\
\begin{enumerate}
    \item Transport:\\
          Die Automobilindustrie verwendet ES mit ABS, ESP, etc., die 
          FLugzeugindustrie benutzt ES f"ur Flugkontrolle, Kollisionsvermeidung,
          Autopiloten, etc. . Auch Schiffe Und Z"uge verwenden ES f"ur Sicherheits
          und Kontroll/ Navigations Features.\\
    \item Logistik:\\
          Die Logistik verwendet ES f"ur Radio-frequency-identification,
          mobile Kommunikation, etc. .\\
    \item Frabriken:\\
          Fabriken verwenden ES in 'social machines`, Maschinen de sich selbst 
          konfigurieren und/ oder distributieren.\\
    \item Structural Safety:\\
          Darunter f"allt Regulation des Wasserstandes eines Damms,
          "Uberwachung von Br"ucken/ Vulkanen, sowie die
          Neigung von Hochhausern bei Erdbeben.\\
    \item Smart Home:\\
          Hauser k"onnen ES f"ur zero energy buildings, safety/ security, 
          comfort, ambient assited living (selbst regulierende Fenster etc.)
          benutzen.\\
\end{enumerate}

\subsection{Nennen Sie mindestens drei Charakteristika bzw. Anforderungen an 
            eingebettete Systeme.}

Anforderungen an ES sind:\\
\begin{enumerate}
    \item Dependability:\\
          Darunter F"allt\\
          \subitem Reliability $R(t)$\\
                   Wahrscheinlichkeit, dass das System funktioniert, wenn es zu 
                   Begin (t=0) funktionierte.\\
          \subitem Maintainability $M(d)$
                   die Wahrscheinlichkeit, dass das System $d$ Zeiteinheiten nach 
                   einem Error funktioniert.\\
          \subitem Availability $A(t)$\\
                   Wahrscheinlichkeit, dass das System zum Zeitpunkt $t$ 
                   funktioniert.\\
          \subitem Safety\\
                   Das System soll dem Umfeld keinen Schaden anrichten k"onnen.\\
          \subitem Security\\
                   Das System soll sicher gegen Attacken au"serhalb des Systems
                   sein.\\
    \item Efficiency:\\
          Bei der Effizienz wird meistens ein Kompromiss zwischen dem best 
          m"oglichem Endprodukt und der einfachsten Herstellungsweise unter den
          folgenden Anforderungen eingegangen (RISC vs. ASIC)\\
          \subitem Code=Gr"o"se/ L"ange\\
          \subitem Laufzeit\\
          \subitem Gewicht der Hardware\\
          \subitem Kosten\\
          \subitem Energieverbrauch\\
    \item Real-Time Constraints:\\
          Nicht alle ES sind Real-Time effizient. Cyber-Physical-Systems (CPS)
          hingegen m"ussen Real-Time effizient sein.\\
          Zu Real-Time constraints k"onnen auch wie folgt aufgefasst werden:\\
          Eine garantierte R"uckmeldung des Systems auf Stimulationen des zu
          kontrollierenden Objekts innerhalb eines Zeitraums, den Das Umfeld
          diktiert.\\
\end{enumerate}
\end{document}
