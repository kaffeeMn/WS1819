\documentclass{article}
\textwidth=6in
\hoffset=0in
\voffset=0in

\usepackage{ngerman}
\usepackage[a4paper, total={6in, 8in}]{geometry}
\usepackage{amsmath}
\usepackage{amssymb}
\usepackage{stmaryrd}
\usepackage{graphicx}
\usepackage{tikz}
\usetikzlibrary{automata, arrows}
\usepackage{pifont}
\usepackage{amssymb}
\usepackage{gensymb}
\usepackage[ampersand]{easylist}

% needs to be updated
\author{Max Springenberg, 177792}
\title{\
    ES "Ubungsblatt 1
    }
\setcounter{section}{1}
\date{}

\begin{document}

\maketitle
\newpage

\subsection{Was ist ein eingebettetes System? Bitte geben Sie keine Definition 
    an, sondern beantworten Sie diese Frage in Ihren eigenen Worten.}

Ein eingebettetes System ist ein System, das Informationen verarbeitet und in 
    ein gr"o"seres Produkt eingebettet ist.\\
Ein eingebettetes System stellt also Software bereit, die durch die Verarbeitung
    von (Sensor-)Signalen/ Informationen ein Produkt um Features, bzw. 
    Verbesserungen erg"anzt.\\

\subsection{Nennen Sie mindestens zwei Anwendungsgebiete eingebetteter Systeme}

Eingebette Systeme(ES) finden Anwendung in:\\
\begin{enumerate}
    \item Transport:\\
          Die Automobilindustrie verwendet ES mit ABS, ESP, etc., die 
          FLugzeugindustrie benutzt ES f"ur Flugkontrolle, Kollisionsvermeidung,
          Autopiloten, etc. . Auch Schiffe Und Z"uge verwenden ES f"ur Sicherheits
          und Kontroll/ Navigations Features.\\
    \item Logistik:\\
          Die Logistik verwendet ES f"ur Radio-frequency-identification,
          mobile Kommunikation, etc. .\\
    \item Frabriken:\\
          Fabriken verwenden ES in 'social machines`, Maschinen de sich selbst 
          konfigurieren und/ oder distributieren.\\
    \item Structural Safety:\\
%Regulation des Wasserstandes eines Damms
%Ueberwachung von Bruecken/ Vilkanen
%Neigung von Hochhausern bei Erdbeben

%Smart Home
%zero energy buildings
%safety/ security
%comfort
%ambient assited living (selbst regulierende Fenster etc.)

%Physical/ Science Experiments
%obeservation of outcomes
\end{enumerate}
\end{document}
