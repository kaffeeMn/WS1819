\documentclass{article}
\textwidth=6in
\hoffset=0in
\voffset=0in

\usepackage{afterpage}
\usepackage{pgf}
\usepackage{tikz}
\usepackage{pdflscape}
\usetikzlibrary{arrows,automata}
\usepackage[latin1]{inputenc}
\usepackage{ngerman}
\usepackage[a4paper, total={6in, 8in}]{geometry}
\usepackage{amsmath}
\usepackage{amssymb}
\usepackage{stmaryrd}
\usepackage{graphicx}
\usepackage{tikz}
\usetikzlibrary{automata, arrows, fit, calc}
\usepackage{pifont}
\usepackage{amssymb}
\usepackage{gensymb}
\usepackage[ampersand]{easylist}


\newcommand{\gap}{\ \\ \\}
\newcommand{\chart}[6]{
    \node[round, 
          minimum width=#5, 
          minimum height=#6] (#1) at (#3,#4) {chart};
    \node[rect] (#2) at (#3,#4) {title};
}
\newcommand{\sepline}[2]{
}

% needs to be updated
\author{Max Springenberg, 177792\\
        Daniel Sonnabend, 190748}
\title{\
    ES "Ubungsblatt 7\\
    Gruppe Fr. 8-10
    }
\setcounter{section}{7}
\date{}

\begin{document}

\maketitle
\\
\gap
\\
\subsection{Vorbereitung}
a) Individuelle Identifikationsmerkmale eines Events sind der Besitzer, 
    der Name (oder die Maske).\\
\\
b) Nur von dem Task, der das Event besitzt ausgef"uhrt werden k"onnen:\\
\begin{enumerate}
    \item ClearEvent\\
    \item WaitEvent\\
\end{enumerate}
\\
c) Nur Extended Tasks k"onnen Besitzer eines Events sein, weil nur diese
    den `waiting state` besitzen, um auf Events zu warten.\\
\\
\end{document}
