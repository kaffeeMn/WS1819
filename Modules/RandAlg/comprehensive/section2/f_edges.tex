\section{$F$-schwere/-leichte Kanten}

$F$-schwere und -leichte Kanten sind ein wesentlicher Bestandteil des 
    MST-Algorithmus. Mittels der Identifizierung von Kanten in einem Graphen $G$
    als $F$-schwer hinsichtlich eines Waldes $F$ in $G$ kann
    bereits entschieden werden, dass diese Kante nicht im MST von $G$ enthalten
    ist. Der umkehrschluss f"ur $F$-leichte Kanten gilt jedoch nicht, wie wir
    sehen werden.\\

\subsection{Definition}

Wir betrachten neben der Gewichtungsfunktion $w$ nun die Funktion $w_F$, mit
    $$
    w_F(\{u,v\}) =  \begin{cases}
                        \infty, P(\{u,v\}) = \emptyset\\
                        max\{w(P_e(\{u,v\}))\}, \text{ sonst}
                    \end{cases}
    $$
, wobei $w(P_e(\{u,v\}))$ bedeutet, dass $w$ auf alle Kanten des Pfades angewandt
    wurde.\\
$w_F$, gibt also das Kantengewicht der Kante mit maximalen Gewicht auf dem
    Pfad von $u$ nach $v$ in $F$ aus. Sollte dieser Pfad nicht existieren, so
    nehmen wir an, dass diese Kante unendlich schwer ist.\\
Ist das Gewicht einer Kante $w(e)$ echt gr"o"ser als das maximale Gewicht auf dem 
    Pfad $P_e(e)$ in $F$, bzw. $w(e) > w_F(e)$, 
    so bezeichnen wir sie als $F$-schwer.
    Sonst ist sie $F$-leicht.

\subsection{Verifikation durch $F$-schwere/-leichte Kanten}

Es liegt nahe, dass $F$ kein MSF in $G$ ist, wenn eine Kante $\{u,v\}$ 
    existiert dessen gewicht echt kleiner als das des maximalen Gewichts auf dem
    Pfad $P_e(\{u,v\})$ in $F$ ist. W"urde man annehmen, dass $F$ ein MSF w"are, 
    so k"onnte man $E_F$ um die Kante $\{u,v\}$ erweitern und den dadurch 
    entstandenen Zyklus mittels entfernen der Kante maximalen Gewichts auf dem
    Pfad $P_e(\{u,v\})$ l"osen. Dadurch h"atte man die Zusammenh"angende 
    Komponente als solche gewahrt und eine schwerere durch eine leichtere Kante
    substituiert. Folglich h"atte man einen leichteren Wald und $F$ w"are damit
    kein MSF.\\
Diese Erkenntnis reicht bereits aus um einen Verifikationsalgorithmus zum 
    MST und MSF Problem zu konstruieren, der lediglich "uber alle Kanten 
    iteriert und schaut ob die einzigen $F$-leichten Kanten die aus $F$ selbst
    sind.\\

\subsection{Informationsgewinn durch $F$-schwere Kanten}

Wir haben bereits erw"ahnt, dass $F$-schwere Kanten nicht in einem MSF, bzw. MST
    enthalten sein k"onnen. 
    Dies l"asst sich durch folgendenen Beweis verdeutlichen.\\
%TODO: ex 10.14 hier
\\
Wir k"onnen also s"amtliche $F$-schwere Kanten f"ur das erfassen des MST 
    ignorieren, bzw. sogar aus $G$ eliminieren.\\
Diese Erkenntnis nimmt sich auch der MST-Algorithmus zum Nuzten. In gewisser
    Hinsicht werden wir den umgekehrten Ansatz von Kruskal, welcher Kanten nach
    gewichten aufsteigend sortiert und immer nur $F$-leichte Kanten hinzunimmt,
    durch das erfassen von $F$-schweren Kanten "uber Stichproben von W"aldern
    in $G$ verfolgen.\\
