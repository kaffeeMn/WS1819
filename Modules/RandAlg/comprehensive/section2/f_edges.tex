\section{$F$-schwere/-leichte Kanten}

$F$-schwere und -leichte Kanten sind ein wesentlicher Bestandteil des 
    MST-Algorithmus. Mittels der Identifizierung von Kanten in einem Graphen $G$
    als $F$-schwer hinsichtlich eines Waldes $F$ in $G$ kann
    bereits entschieden werden, dass diese Kante nicht im MST von $G$ enthalten
    ist. Der Umkehrschluss f"ur $F$-leichte Kanten gilt jedoch nicht, wie wir
    sehen werden.
    Betrachten wir also eine Approximation $F$ eines MST bez"uglich $G$, so k"onnen
    wir neben dem Gewicht des Waldes auch die Anzahl von $F$-leichten Kanten in $G$ als
    G"utema"s verwenden.\\

\subsection{Definition}

Wir betrachten neben der Gewichtungsfunktion $w$ nun die Funktion $w_F$, mit
    $$
    w_F(\{u,v\}) =  \begin{cases}
                        \infty, P(\{u,v\}) = \emptyset\\
                        max\{w(P_e(\{u,v\}))\}, \text{ sonst}
                    \end{cases}
    $$
, wobei $w(P_e(\{u,v\}))$ bedeutet, dass $w$ auf alle Kanten des Pfades angewandt
    wurde.\\
$w_F$, gibt also das Kantengewicht der Kante mit maximalen Gewicht auf dem
    Pfad von $u$ nach $v$ in $F$ aus. Sollte dieser Pfad nicht existieren, so
    nehmen wir an, dass diese Kante unendlich schwer ist.\\
Ist das Gewicht einer Kante $w(e)$ echt gr"o"ser als das maximale Gewicht auf dem 
    Pfad $P_e(e)$ in $F$, bzw. $w(e) > w_F(e)$, 
    so bezeichnen wir sie als $F$-schwer.
    Sonst ist sie $F$-leicht.

\subsection{Informationsgewinn durch $F$-schwere Kanten}

Wir haben bereits erw"ahnt, dass $F$-schwere Kanten nicht in einem MSF, bzw. MST
    enthalten sein k"onnen.
    Dies werden wir im folgenden beweisen.
    Aufbauend darauf k"onnen wir dann einen Verifikationsalgorithmus definieren.\\

\subsubsection{Beweis}
\label{sec:fProof}

Sei $F$ ein beliebiger Baum in $G$.
    Existiert eine $F$-schwere Kante $e=\{u, v\}$ in $G$, so gelten folgende 
    Eigenschaften f"ur $F$:\\
(i) Es existiert ein Pfad zwischen $u$ und $v$\\
(ii) $\forall e' \in P_e(e) : w(e') < w(e)$\\
W"are $e$ im $MSF$ von $G$ enthalten, so w"urde das Tauschen einer Kante aus
    $F$ durch $e$ die Approximation $F$ verbessern.\\
Aus (i) folgt, dass durch $e$ ein Zyklus entsteht. Insbesondere bedeutet das, 
    dass wir f"ur $e$ eine Kante auf dem Pfad zwischen $u$ und $v$ tauschen
    m"ussten.
    Aus (ii) folgt jedoch, dass dies $F$ verschlechtern w"urde.
    Damit kann $e$ nicht im MSF von $G$ enthalten sein.\\
Der Umkehrschluss, dass alle $F$-leichten Kanten im MSF vorkommen gilt jedoch
    nicht.
    Betrachten wir beispielsweise einen vollst"andigen Graphen $G_{w_1}$ mit der 
    Gewichtungsfunktion $w(e) = 1, \forall e \in E_{G_{w_1}}$ und 
    $n = |V_{G_{w_1}}| > 2$, so stellen wir fest, dass jeder Pfad in $G$, der alle
    Knoten verbindet ein MST $F$ von $G$ ist.
    Zudem ist jede Kante $F$-leicht, da alle Kanten gleich gewichtet werden.
    W"are jede $F$-leichte Kante im MST enthalten, so w"are der MST gleich $G$,
    da $G$ vollst"andig und $n > 2$ ist, w"are aber mindestens ein Zyklus im 
    MST enthalten und damit eine Baum-, bzw. Wald-Eigenschaft verletzt.\\
    Damit gilt der Umkehrschluss nicht.\\
\\
Wir k"onnen also s"amtliche $F$-schwere Kanten f"ur das erfassen des MST 
    ignorieren, bzw. sogar aus $G$ eliminieren.\\
Diese Erkenntnis nimmt sich auch der MST-Algorithmus zum Nuzten.\\

\subsubsection{Verifikation durch $F$-schwere/-leichte Kanten}
\label{sec:verification}
Es liegt nahe, dass $F$ kein MSF in $G$ ist, wenn eine Kante $\{u,v\}$ 
    existiert dessen gewicht echt kleiner als das des maximalen Gewichts auf dem
    Pfad $P_e(\{u,v\})$ in $F$ ist. W"urde man annehmen, dass $F$ ein MSF w"are, 
    so k"onnte man $E_F$ um die Kante $\{u,v\}$ erweitern und den dadurch 
    entstandenen Zyklus mittels entfernen der Kante mit maximalen Gewicht auf dem
    Pfad $P_e(\{u,v\})$ l"osen. Dadurch h"atte man die Zusammenh"angende 
    Komponente als solche gewahrt und eine schwerere durch eine leichtere Kante
    substituiert. Folglich h"atte man einen leichteren Wald und $F$ w"are damit
    kein MSF.\\
Diese Erkenntnis reicht bereits aus um einen Verifikationsalgorithmus zum 
    MST und MSF Problem zu konstruieren.
    So k"onnte man auf $F$ eine Tiefensuche durchf"uhren und f"ur jeden Pfad 
    das maximale Kantengewicht unter dem Start und Endknoten dessen mittels
    Hashing in einer Hashmap $w_F : E \rightarrow \mathbb{R}$ abspeichern die 
    $\infty$ ausgibt, wenn f"ur eine Kante kein Wert gesetzt wurde.
    Anschlie"send w"urde man "uber $E_G$ iterieren und sicherstellen, dass gilt
    $\forall e \in E_G: w(e) \leq w_F(e)$.\\
Man kann den Verfikationsalgorithmus auch so anpassen, dass die F-schweren Kanten
    ausgegeben werden.
    Ferner l"auft der Algorithmus in Linearzeit.\\
