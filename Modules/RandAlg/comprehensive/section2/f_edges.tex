\section{$F$-schwere/-leichte Kanten}
\label{sec:fEdg}

$F$-schwere und -leichte Kanten sind ein wesentlicher Bestandteil des 
    MST-Algorithmus. Mittels der Identifizierung von Kanten in einem Graphen $G$
    als $F$-schwer hinsichtlich eines Waldes $F$ in $G$ kann
    bereits entschieden werden, dass diese Kante nicht im MST von $G$ enthalten
    ist. Der Umkehrschluss f"ur $F$-leichte Kanten gilt jedoch nicht, wie wir
    sehen werden.
    Betrachten wir also eine Approximation $F$ eines MST bez"uglich $G$, so k"onnen
    wir neben dem Gewicht des Waldes auch die Anzahl von $F$-leichten Kanten in $G$ als
    G"utema"s verwenden.\\

\subsection{Definition}

Wir betrachten neben der Gewichtungsfunktion $w$ nun die Funktion $w_F$, mit
    $$
    w_F(\{u,v\}) =  \begin{cases}
                        \infty, P_e(\{u,v\}) = \emptyset\\
                        max\{w(P_e(\{u,v\}))\}, \text{ sonst}
                    \end{cases}
    $$
, wobei $w(P_e(\{u,v\}))$ bedeutet, dass $w$ auf alle Kanten des Pfades angewandt
    wurde. Hierbei und im Folgenden bezieht sich $P_e$ auf Kantenfolgen von 
    Pfaden in dem betrachteten Wald, bzw. Baum.\\
$w_F$, gibt also das Kantengewicht der Kante mit maximalem Gewicht auf dem
    Pfad von $u$ nach $v$ in $F$ aus. Sollte dieser Pfad nicht existieren, so
    nehmen wir an, dass diese Kante unendlich schwer ist.\\
Ist das Gewicht $w(e)$ einer Kante $e$ echt gr"o"ser als das maximale Gewicht auf dem 
    Pfad $P_e(e)$ in $F$, bzw. $w(e) > w_F(e)$, 
    so bezeichnen wir sie als $F$-schwer.
    Sonst ist sie $F$-leicht.

\subsection{Informationsgewinn durch $F$-schwere Kanten}

Wir haben bereits erw"ahnt, dass $F$-schwere Kanten nicht in einem MSF, bzw. MST
    enthalten sein k"onnen.
    Dies werden wir im Folgenden beweisen
    und aufbauend darauf dann einen Verifikationsalgorithmus definieren.\\

\subsubsection{Beweis}
\label{sec:fProof}

Sei $F$ ein MSF in $G$.
    Existiert eine $F$-schwere Kante $e=\{u, v\}$ in $G$, so gelten folgende 
    Eigenschaften f"ur $F$:\\
(i) Es existiert ein Pfad zwischen $u$ und $v$ in $F$, mit $e \not \in P_e(e)$,
    da $w(e) > w_F(e)$\\
(ii) $\forall e' \in P_e(e) : w(e') < w(e)$\\
W"are $e$ in einem $MSF$ von $G$ enthalten, so w"urde das Tauschen einer Kante aus
    $F$ durch $e$ das Gewicht des MSF $F$ nicht vergr"o"sern.
    Nehmen wir also an, das $e$ in einem $MSF$ von $G$ enthalten sei.\\
Aus (i) folgt, dass durch Hinzunahme von $e$ ein Zyklus entsteht. 
    Insbesondere bedeutet das, 
    dass wir f"ur $e$ eine Kante auf dem Pfad zwischen $u$ und $v$ austauschen
    m"ussten.
    Aus (ii) folgt jedoch, dass dies $F$ verschlechtern w"urde.
    Dies erzeugt einen Widerspruch zur Annahme.
    Somit kann $e$ nicht in einem MSF und damit auch nicht in einem MST
    von $G$ enthalten sein.\\
Der Umkehrschluss, dass alle $F$-leichten Kanten im MSF vorkommen gilt jedoch
    nicht.
    Nehmen wir an, dass jede $F$-leichte Kante eines Graphen auch in einem
    MSF enthalten sein muss.
    Betrachten wir nun einen vollst"andigen Graphen $G_{w_1}$ mit der 
    Gewichtungsfunktion $w(e) = 1, \forall e \in E_{G_{w_1}}$ und 
    $n_{G_{w_1}} > 2$, so stellen wir fest, dass man aus jedem Pfad $P$ in $G$, 
    der alle
    Knoten verbindet einen MST $F$ von $G$ mit $F=(P, P_e)$ 
    bilden kann.
    Zudem ist jede Kante $F$-leicht, da alle Kanten gleich gewichtet werden.
    W"are jede $F$-leichte Kante im MST enthalten, so w"are der MST gleich $G$.
    Da $G$ vollst"andig und $n_{G_{w_1}} > 2$ gilt, w"are aber mindestens ein Zyklus im 
    MST enthalten und damit eine Baum-, bzw. Wald-Eigenschaft verletzt.\\
    Dies erzeugt einen Widerspruch zur Annahme.
    Somit gilt der Umkehrschluss nicht.\\
\\
Wir k"onnen also s"amtliche $F$-schwere Kanten f"ur das Erfassen des MST 
    ignorieren, bzw. sogar aus $G$ eliminieren.\\
Diese Erkenntnis nimmt sich auch der MST-Algorithmus zum Nutzen.\\

\subsubsection{Verifikation durch $F$-schwere/-leichte Kanten}
\label{sec:verification}
Es liegt nahe, dass $F$ kein MSF in $G$ ist, wenn eine Kante $\{u,v\}$ 
    existiert, dessen Gewicht echt kleiner als das des maximalen Gewichts auf dem
    Pfad $P_e(\{u,v\})$ in $F$ ist. 
    N"ahme man an, dass $F$ ein MSF w"are, 
    so k"onnte man $E_F$ um die Kante $\{u,v\}$ erweitern und den dadurch 
    entstandenen Zyklus mittels Eliminieren der Kante mit maximalem Gewicht auf dem
    Pfad $P_e(\{u,v\})$ in $F$ l"osen. Dadurch h"atte man die zusammenh"angende 
    Komponente als solche gewahrt und eine schwerere durch eine leichtere Kante
    substituiert. Folglich h"atte man einen Wald mit geringerem Gewicht
    und $F$ w"are damit kein MSF.\\
Diese Erkenntnis reicht bereits aus, um einen Verifikationsalgorithmus zum 
    MST und MSF Problem zu konstruieren.
    Im folgenden wird ein einfacher Algorithmus geschildert, 
    der zeigt, wie $w_F$ f"ur einen
    Graphen konstruiert werden kann.
    Man k"onnte "uber jeden Pfad in $F$ iterieren und f"ur diesen 
    das maximale Kantengewicht unter dem Start und Endknoten dessen mittels
    Hashing und einer geeigneten Hash-Funktion in einer Hashmap 
    $h_{w_F} : E \rightarrow \mathbb{R}$ abspeichern.
    Die Hashmap soll ferner $\infty$ ausgeben, wenn f"ur eine Kante kein Wert 
    gesetzt wurde.
    Dadurch h"atte man $w_F$ mittels $h_{w_F}$ konstruiert.
    Anschlie"send w"urde man "uber $E_G$ iterieren und sicherstellen, dass gilt
    $\forall e \in E_G: w(e) \leq w_F(e)$.\\
    Im Worst-Case liegen Alle Knoten aus $F$ auf einem Pfad.
    In diesem Fall w"urden $\sum_{i=1}^{n_F} i = \frac{n_F^2+n_F}{2}$ Pfade existieren.
    F"ur dichte Graphen $G$ mit $m_G \rightarrow n_G^2 = n_F^2$ w"are dies sogar noch im Rahmen 
    einer linearen Laufzeit, im allgemeinen jedoch nicht.
    Es gibt allerdings deterministische Algorithmen, wie den von V. King 
    \cite{simpleVer}
    die in Linearzeit auf beliebigen Graphen verifizieren.
    Eine Aufbereitung dieses Algorithmus w"urde jedoch den Rahmen dieser 
    Ausarbeitung sprengen.\\
Man kann den Verifikationsalgorithmus auch so anpassen, dass die F-schweren Kanten
    ausgegeben werden.\\
