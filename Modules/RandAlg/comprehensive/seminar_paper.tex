%\documentclass{article}
\documentclass[a4paper,12pt,times,german]{cls/summary}
\usepackage{setspace}
\onehalfspacing

\usepackage{ngerman}
\usepackage{amsmath}
\usepackage{amssymb}
\usepackage{algorithmic}
\usepackage[]{algorithm2e}
\usepackage{float}
\restylefloat{table}
\usepackage{hyperref}


\title{Der Linearzeit MST-Algorithmus\\
       \LARGE Der Schnellste Algorithmus F"ur Das MST/ MSF Problem}
\author{Maximilian Springenberg\\
        \small Proseminar Randomisierte Algorithmen}
\date{}
\setcounter{section}{-1}

\begin{document}
\maketitle

% paragraphs
\section{Konventionen}
In dieser Ausarbeitung wird sich an den "ublichen Konventionen zur Notation
    von Variablen in Graphen orientiert. So definieren wir einen Graphen als
    $G = (V,E)$, mit der Anzahl von Knoten $n = |V|$ und Kanten $m = |E|$.
    Zu $G$ zugeh"orige Kanten und Knotenmengen werden in dieser Ausarbeitung
    auch durch $V_G$, $E_G$ und ihre M"achtigkeiten durch $n_G, m_G$
    als solche gekennzeichnet.
    F"ur jeden Knoten $v$ nehmen wir an, dass eine Adjazenzliste $adj(v)$ von
    adjazenten Knoten f"ur ihn vorhanden ist.
    Ferner betrachten wir ungerichtete gewichtete Graphen und bezeichnen 
    die Gewichtungsfunktion als $w: E \rightarrow \mathbb{R}$.\\
Des weiteren werden auch Wege, bzw. Pfade betrachten. 
    Dazu sei $P(\{v,u\})$ eine Funktion, die in einem Baum den Pfad 
    von $v$ nach $u$ als Knotenfolge angibt.
    Da es f"ur uns handlicher sein wird
    gleich mit den Kanten zu arbeiten, definieren wir $P_e$ als die Folge von
    Kanten auf dem Pfad.\\

%!TEX root = ../compehension.tex
\section{Motivation}

\subsection{Konventionen}
In dieser Ausarbeitung wird sich an die "ublichen Konventionen zur Notation
    von Variablen in Graphen orientiert. So definieren wir einen Graphen als
    $G = (V,E)$, mit der Anzahl von Knoten $n = |V|$ und Kanten $m = |E|$.
    Ferner betrachten wir ungerichtete gewichtete Graphen und bezeichnen 
    die Gewichtungsfunktion als $w: E \rightarrow \mathbb{R}$.\\

\subsection{MST und MSF}
Der minimale Spannbaum, oder auch MST, stellt einen azyklischen 
    zusammenh"angenden Teilgraph aus G, der alle Knoten verbindet und
    dessen Summe von Kantengewichte $\sum_{e \in E_{MST}} w(e)$
    minimal ist dar.\\
Ist G selbst nicht zusammenh"angend, so werden wir einen minimalen Spannwald
    als n"achst beste L"osung betrachten.
    Ein minimaler Spannwald besteht aus einer Sammlung von minimalen 
    Spannb"aumen f"ur die verbundenen Komponenten in G.\\

\subsection{Warum ein nichtdeterministischer Anzatz von Vorteilen sein kann}

Wir werden in dieser Ausarbeitung zum Kapitel 10.3 aus dem Buch
    `Randomized Algorithms` von Motwani, R., Raghavan, P.  
    einen
    randomisierten Ansatz f"ur einen Algorithmus, der einen minimalen Spannbaum
    in erwarteter Linearzeit aproximiert betrachten.\\
Das MST Problem ist in $P$, und es sind bereits 
    deterministische Algorithmen wie die von
    Prim, Kruskal, Bor\r uvka bekannt
    , die mit einer Worst-Case Laufzeitschranke 
    von $O(m * log(n))$ das Problem l"osen.
    Zudem existiert der Algorithmus von Bernard Chazelle, f"ur den eine
    Worst-Case Laufzeitschranke von $O(m * log \beta(m,n))$ bekannt ist, wobei
    mit
    $\beta(m,n) = \{i | log^{(i)} n \leq m / n\}$ die inverse Ackermann Funktion
    verwendet wird. 
    $log^{(i)} n$ ist hierbei die $i$-te Anwendung von $log$ auf $n$.
    Dementsprechend steigt die Funktion $\beta$ so schwach, dass die
    aus ihr resultierenden Faktoren f"ur die Worst-Case Laufzeit hinsichtlich
    der Gr"o"se von Graphen in der Praxis als nahezu konstant angesehen werden.\\
Wozu dient also ein nichtdeterministischer Algorithmus, der im Erwartungswert 
    linear verl"auft?
    - Die Antwort auf diese Frage kann sowohl durch die komplexe Implementierung
    des Algorithmus von Chazelle, als auch der Stabilit"at, bzw. G"ute,
    des vorgestellten Algorithmus hinsichtlich seiner Laufzeit
    und nicht zuletzt durch das Betrachten des Algorithmus als akademisches
    Beispiel f"ur kreative Laufzeitverbesserungen begr"undet werden.\\

\section{$F$-schwere/-leichte Kanten}

$F$-schwere und -leichte Kanten sind ein wesentlicher Bestandteil des 
    MST-Algorithmus. Mittels der Identifizierung von Kanten in einem Graphen $G$
    als $F$-schwer hinsichtlich eines Waldes $F$ in $G$ kann
    bereits entschieden werden, dass diese Kante nicht im MST von $G$ enthalten
    ist. Der Umkehrschluss f"ur $F$-leichte Kanten gilt jedoch nicht, wie wir
    sehen werden.
    Betrachten wir also eine Approximation $F$ eines MST bez"uglich $G$, so k"onnen
    wir neben dem Gewicht des Waldes auch die Anzahl $F$-leichten Kanten als
    G"utema"s verwenden.\\

\subsection{Definition}

Wir betrachten neben der Gewichtungsfunktion $w$ nun die Funktion $w_F$, mit
    $$
    w_F(\{u,v\}) =  \begin{cases}
                        \infty, P(\{u,v\}) = \emptyset\\
                        max\{w(P_e(\{u,v\}))\}, \text{ sonst}
                    \end{cases}
    $$
, wobei $w(P_e(\{u,v\}))$ bedeutet, dass $w$ auf alle Kanten des Pfades angewandt
    wurde.\\
$w_F$, gibt also das Kantengewicht der Kante mit maximalen Gewicht auf dem
    Pfad von $u$ nach $v$ in $F$ aus. Sollte dieser Pfad nicht existieren, so
    nehmen wir an, dass diese Kante unendlich schwer ist.\\
Ist das Gewicht einer Kante $w(e)$ echt gr"o"ser als das maximale Gewicht auf dem 
    Pfad $P_e(e)$ in $F$, bzw. $w(e) > w_F(e)$, 
    so bezeichnen wir sie als $F$-schwer.
    Sonst ist sie $F$-leicht.

\subsection{Informationsgewinn durch $F$-schwere Kanten}

Wir haben bereits erw"ahnt, dass $F$-schwere Kanten nicht in einem MSF, bzw. MST
    enthalten sein k"onnen.
    Dies werden wir im folgenden Beweisen.
    Aufbauend darauf k"onnen wir dann einen Verifikationsalgorithmus definieren.\\

\subsubsection{Beweis}
\label{sec:fProof}

Sei $F$ ein beliebiger Baum in $G$.
    Existiert eine $F$ schwere Kante in $G$, $e=\{u, v\}$, so gelten folgende 
    Eigenschaften f"ur $F$:\\
(i) Es existiert ein Pfad zwischen $u$ und $v$\\
(ii) Die Gewichtung jeder Kante auf dem Pfad ist leichter als $w(e)$\\
W"are $e$ im $MSF$ von $G$ enthalten, so w"urde das Tauschen einer Kante aus
    $F$ durch $e$ die Approximation $F$ verbessern.\\
Aus (i) folgt, dass durch $e$ ein Zyklus entsteht. Insbesondere bedeutet das, 
    dass wir f"ur $e$ eine Kante auf dem Pfad zwischen $u$ und $v$ tauschen
    m"ussten.
    Aus (ii) folgt jedoch, dass dies $F$ verschlechtern w"urde.
    Damit kann $e$ nicht im MSF von $G$ enthalten sein.\\
Der Umkehrschluss, dass alle $F$-leichten Kanten im MSF vorkommen gilt jedoch
    nicht.
    Betrachten wir beispielsweise einen vollst"andigen Graphen $G_{w_1}$ mit der 
    Gewichtungsfunktion $w(e) = 1, \forall e \in E_{G_{w_1}}$ und 
    $n = |V_{G_{w_1}}| > 2$, so stellen wir fest, dass jeder Pfad in $G$, der alle
    Knoten verbindet ein MST $F$ von $G$ ist.
    Zudem ist jede Kante $F$-leicht, da alle Kanten gleich gewichtet werden.
    W"are jede $F$-leichte Kante im MST enthalten, so w"are der MST gleich $G$,
    da $G$ vollst"andig und $n > 2$ ist, w"are aber mindestens ein Zyklus im 
    MST enthalten und damit eine Baum-Eigenschaft verletzt.\\
    Damit gilt der Umkehrschluss nicht.\\
\\
Wir k"onnen also s"amtliche $F$-schwere Kanten f"ur das erfassen des MST 
    ignorieren, bzw. sogar aus $G$ eliminieren.\\
Diese Erkenntnis nimmt sich auch der MST-Algorithmus zum Nuzten. In gewisser
    Hinsicht werden wir den umgekehrten Ansatz von Kruskal, welcher Kanten nach
    gewichten aufsteigend sortiert und immer nur $F$-leichte Kanten hinzunimmt,
    durch das erfassen von $F$-schweren Kanten "uber Stichproben von W"aldern
    in $G$ verfolgen.\\

\subsubsection{Verifikation durch $F$-schwere/-leichte Kanten}
\label{sec:verification}
Es liegt nahe, dass $F$ kein MSF in $G$ ist, wenn eine Kante $\{u,v\}$ 
    existiert dessen gewicht echt kleiner als das des maximalen Gewichts auf dem
    Pfad $P_e(\{u,v\})$ in $F$ ist. W"urde man annehmen, dass $F$ ein MSF w"are, 
    so k"onnte man $E_F$ um die Kante $\{u,v\}$ erweitern und den dadurch 
    entstandenen Zyklus mittels entfernen der Kante maximalen Gewichts auf dem
    Pfad $P_e(\{u,v\})$ l"osen. Dadurch h"atte man die Zusammenh"angende 
    Komponente als solche gewahrt und eine schwerere durch eine leichtere Kante
    substituiert. Folglich h"atte man einen leichteren Wald und $F$ w"are damit
    kein MSF.\\
Diese Erkenntnis reicht bereits aus um einen Verifikationsalgorithmus zum 
    MST und MSF Problem zu konstruieren.
    So k"onnte man auf $F$ eine Tiefensuche durchf"uhren und f"ur jeden Pfad 
    das maximale Kantengewicht unter dem Start und Endknoten dessen mittels
    Hashing in einer Hashmap $w_F : E_F \rightarrow \mathbb{R}$ abspeichern die 
    $\infty$ ausgibt, wenn f"ur eine Kante kein Wert gesetzt wurde.
    Anschlie"send w"urde man "uber $E_G$ iterieren und sicherstellen, dass gilt
    $\forall e \in E_G: w(e) \leq w_F(e)$.\\
Man den Verfizierungsalgorithmus auch so anpassen, dass die F-schweren Kanten
    ausgegeben werden.
    Ferner l"auft der Algorithmus in linearzeit.\\

%!TEX root = ../comprehension.tex
\section{Reduzieren des Graphen}

Wir werden einen rekursiven Algorithmus konstruieren. 
    Die Rekursion terminiert, wenn nach einer Bor\r uvka Phase der Graph 
    unver"andert bleibt, oder G leer ist.
    Folglich m"ussen wir
    bei jedem rekursiven Aufruf unseren Graphen reduzieren. 
    Wir werden in 
    diesem Teil betrachten, wie wir durch Verwendung von Bor\r uvka-Phasen die
    Knoten des Graphen reduzieren k"onnen und anschlie"send durch randomisierte
    Stichproben zuz"uglich eine Methode kennen lernen um die Anzahl von Kanten
    zu reduzieren. 
    Letzteres erm"oglicht uns erst eine erwartete lineare 
    Laufzeit, da die Rekursiontiefe durch das alleinige reduzieren von Knoten
    f"ur unseren Anspruch zu gro"s ist.\\

%!TEX root = ./graph_reduction.tex
\subsection{Bor\r uvka-Phasen}

\subsubsection{Idee}
\label{sec:borIdea}

Bor\r uvka-Phasen beruhen auf der Erkenntnis, dass in einem beliebigen 
    ungerichteten Graphen $G$ f"ur jeden Knoten $v \in V$ die inzidente Kante mit 
    minimaler Gewichtung 
    $e_{v^{min}} := \{v, u\}, u \in adj(v): 
        \nexists e' = \{v, u'\}, u' \in adj(v): w(e') < w(e)$
    im MST von $G$ enthalten ist.\\
Hierf"ur k"onnen wir einen beliebigen Knoten $v \in V_G$ und die zu $v$ inzidente
    Kante mit minimalem Gewicht $e_{min}$ in einem 
    zusammenh"angenden Graphen $G$ betrachten.\\
1. Fall $adj(v)$ enth"alt nur eine Kante oder mehrere inzidenten Kanten sind minimal. 
    Dann ist die minimale Kante Kante zwangsweise im 
    MST, oder ein MST mit gleichem Gewicht kann durch eine der anderen minimalen
    Kanten knostruiert werden.\\
2. Fall $|adj(v)| > 1$ und nur eine Kante ist minimal. 
    Nehmen wir an, dass ein MST existiert, der $e_{min}$ 
    nicht enth"alt. 
    Der MST verbindet alle Knoten.
    Entfernen wir die zu $v$ inzidente Kante im MST
    und ersetzen wir sie durch $e_{min}$, so haben wir keinen Zyklus und das
    Gewicht f"ur des MST veringert. 
    Damit war der MST nicht minimal.
    Aus dem Widerspruch folgt, dass die minimale inzidente Kante eines jeden
    Knoten im MST enthalten sein muss.\\
\\
Des weiteren ist f"ur uns interessant, dass durch das Markieren der minimalen
    Kanten kein Zyklus entsteht. 
    Damit ein Zyklus entstehen w"urde m"usste ein Pfad geschlossen werden, 
    indem mindestens eine zus"atzliche Kante hinzugenommen wird. 
    Dessen Kantengewicht m"usste kleiner, als das der bereits inzidenten, 
    minimalen, markierten Kante sein.\\
Ferner bedeutet das, dass wir durch die f"ur die Kontraktion markierten Kanten
    $E_{min}$ nach einer Bor\r uvka-Phase einen Wald $F$ in $G$ induzieren.
    Dies wird f"ur uns insebesondere dann interessant, wenn
    wir rekursiv einen Wald aufbauen m"ochten.\\
\\
Eine Bor\r uvka-Phase hat also den folgenden Ablauf:
\begin{enumerate}
    \item Markiere inzidente Kanten $E_{min}$ mit minimaler Gewichtung
    \item Bestimme die verbundenen Komponenten in $G' = (V,E_{min})$
    \item Ersetze jede verbundene Komponente durch einen sie repr"asentierenden
          Knoten in $G'$ und erhalte den Graphen $G''$
    \item Entferne alle Selbstschleifen in $G''$
\end{enumerate}

\subsubsection{Reduktion der Knoten}
Es werden die durch $E_{min}$ verbundenen Komponenten auf je einen Knoten 
    reduziert.
    Wir interssieren uns f"ur den Worst-Case, also die maximale Anzahl an 
    Komponenten.
    Die kleinste Verbundene Komponente w"are theoretisch ein
    einzelner Knoten ohne Kanten.
    An dieser Stelle w"urde aber unser Algorithmus terminieren.
    Die kleinste Verbundene Komponente die wir in einer Bor\r uvka-Phase 
    betrachten werden besteht also aus zwei Knoten und einer Kante.
    Da $G$ zusammeng"angend ist existieren damit maximal $n/2$ verbundene
    Komponenten.\\
Wir stellen fest, dass eine Bor\r uvka-Phase die Menge an Knoten in $G$ auf 
    maximal die H"alfte reduziert, n"amlich genau dann, wenn die markierten 
    Kanten ein perfektes Matching beim bestimmen der verbundenen Komponenten
    induzieren.\\

%!TEX root = ./graph_reduction.tex
\subsection{Randomisierte Stichproben}


\section{Der MST-Algorithmus}

Nun da wir Verfahren zum reduzieren von Knoten und Kanten des Graphen $G$ kennen
    gelernt haben  k"onnen wir anfangen einen Algorithmus zu konstruieren,
    dessen Ausgabe ein Wald $F$ in $G$ ist.\\
Die erste Entscheidung die wir treffen m"ussen, ist ob wir zuerst Knoten oder Kanten
    reduzieren. \\
Da wir in \hyperref[sec:borIdea]{\textit{3.1.1}} gelernt haben, dass
    Bor\r uvka-Phasen nur Kanten markieren, die im MST/MSF 
    enthalten sind fangen wir mit Bor\r uvka-Phasen an und erhalten den Graphen 
    $G_1$.
    Dadurch veringern wir
    zun"achst noch deterministisch die Anzahl $F$-leichter Kanten in $G$.
    Sollte wir in in den Bor\r uvka-Phasen terminieren geben wir den Teilgraph
    mit allen markierten Kanten aus.
    Die Bor\r uvka Phasen werden also auch das Abbruchkriterium unserer
    Rekursion sein.\\
Anschlie"send konstruieren wir $G_2 = G_1(p)$ und f"uhren einen rekursiven
    Aufruf auf diesem durch. 
    Wir wissen also durch \hyperref[sec:goodnessRand]{\textit{3.2.1}}, 
    dass $G$ im Erwartungswert
    $\frac{n/2}{p}$ $F_2$-leichte Kanten vor dem rekursiven Aufruf enth"alt.\\
Wir haben in \hyperref[sec:verification]{\textit{2.2.2}} einen 
    Verfikationsalgorithmus definiert, der auch $F_2$-schwere Kanten ausgeben
    kann.
    Da wir durch \hyperref[sec:fProof]{\textit{2.2.1}} wissen, dass $F_2$-schwere Kanten nicht im MST/MSF von $G$
    enthalten sein k"onnen, entfernen wir alle $F_2$-schwere Kanten $E_{F_2-heavy}$
    aus $G_1$,
    auf dem lediglich die Bor\r uvka-Phasen angewandt wurden und erhalten $G_3$.\\
F"uhren wir nun nocheinmal einen rekursiven Aufruf auf $G_3$ durch, 
    so erhalten wir 
    einen Wald $F_3$ mit intuitiv weniger $F_3$-schweren, als $F_2$-schweren Kanten in $G$.
    Ferner ist insbesondere in dichten Graphen zu erwarten, dass die Stichproben 
    auf $G_3$ nur unwahrscheinlich die Anzahl von Verbundenen Komponenten in 
    $G_3$ um ein wesentliches erh"ohen, bzw. Komponenten aus $G_3$ nicht mehr
    zusammen h"angen.\\
Nachdem wir den Wald $F_3$ bestimmt haben geben wir unsere, von den Bor\r uvka-Phasen
    markierten, Kanten in $C$ vereinigt mit $F_3$ aus.\\

\begin{algorithm}
\KwData{Graph $G$}
\KwResult{Approximation eines MST/ MSF in $G$}
\begin{algorithmic}[1]
    \STATE $G_1, C$ $\leftarrow$\begin{tabular}[H]{l}
                                 3 Bor\r uvka-Phasen werden auf $G$ angewandt.
                                 Dabei wird der resultierende Graph und\\
                                 ein Teilgraph $C$ mit den zur
                                 Kokatenierung markierten Kanten zur"uck gegeben.\\
                                 \textbf{Wenn} $G$ leer ist oder in den Bor\r uvka-Phasen
                                 terminiert wird geben wir $C$ aus.\\
                                 \end{tabular}
    \STATE $G_2$ $\leftarrow$ $G_1(p=0,5)$
    \STATE $F_2$ $\leftarrow$ $MST(G_2)$
    \STATE $G_3$ $\leftarrow$ $(V_{G_1}, E_{G_1} - E_{F_2-heavy})$
    \STATE $F_3$ $\leftarrow$ $MST(G_3)$
    \RETURN $C \cup F_3$
\end{algorithmic}
\end{algorithm}

\subsection{Laufzeit}

$T(n,m)$ sei die erwartete Laufzeit des MST-Algorithmus f"ur einen Graphen $G$.
    Zeile 1 benutzt 3 Bor\r uvka-Phasen und  l"auft damit in $O(n+m)$. 
    $G_2 = G_1(p)$ aus Zeile 2 kann ebenfalls in $O(m + n)$ berechnet werden.
    $G_1$ hat nur noch $n/2^3 = n/8$ Knoten und f"ur $G_2$ hat damit 
    $|V_{G_2}| = n/8$ Knoten und im Erwartungswert $|E_{G_2}| = m/2$ Kanten.
    Folglich ben"otigt die Berechnung von $F_3$  die erwartete Laufzeit 
    $T(n/8,m/2)$.
    Die Berechnugn $F_2$-schwerer Kanten und die Konstruktion von $G_3$ mittels
    derer ben"otigt ebenfalls $O(m+n)$.
    Nach dem Lemma 10.19 \cite{randAlg} ist die Anzahl von $F$-leichten Kanten in $G$ durch
    $n/p = 2n,$ mit $p = 0,5$ im Erwartungswert gegeben. Da $G_1$
    aber nur $n/8$ Knoten hat, k"onnen wir die Anzahl an Kanten in $G_3$
    durch $2*n/8 = n/4$ im Erwartungswert absch"atzen.
    Damit bel"auft sich die erwartete Laufzeit der Berechnung von $F_3$ auf
    $T(n/2, n/4)$.
    Die Vereinigung von $C$ und $F$ in Zeile 6 ben"otigt $O(n)$.\\
Aus den Laufzeiten der Teilschritte folgt
    $T(n,m) \leq T(n/8, m/2) + T(n/8, n/4) + c(n+m)$.
    Nach der Literatur kann $T(n,m)$ durch $2c(n+m)$ abgesch"atzt werden.
% TODO: Kuerzer
    Ein Beweis, dass $T(n,m)$ tats"achlich in $O(n+m)$ enthalten ist befindet
    sich im Anhang.\\
    \\


% literature
\begin{thebibliography}{}
\footnotesize
\bibitem{randAlg} 
    Motwani, R., Raghavan, P. :
    \textit{Randomized Algorithms}. Cambridge :
    Cambridge University Press 1995, Kapitel 10.3.

\bibitem{simpleVer} 
    King, V.:
    \textit{A Simpler Minimum Spanning Tree Verification Algorithm}:
    Algorithmica June 1997
\end{thebibliography}

%% appendix
%\newpage
%\section{Anhang}
\begin{flalign*}
    T(n,m) \leq\ & T(n/8, m/2) + T(n/8, n/4) + c(n+m)&&\\
           \leq\ & (T(n/8^2, m/2^2) + T(n/8^2, \frac{m/2}{4}) + c(n/8+m/2))&&\\
                 & + (T(n/8^2, \frac{n/4}{2}) + T(n/8^2, n/4^2) + c(n/8+n/4))&&\\
                 & + c(n+m)&&\\
           =   \ & (T(n/8^2, m/2^2) + T(n/8^2, m/2^3) + c(n/8+m/2))&&\\
                 & + (T(n/8^2, n/2^3) + T(n/8^2, n/2^4) + c(n/8+n/2^2))&&\\
                 & + c(n+m)&&\\
           \leq\ & (T(n/8^3, m/2^3) + T(n/8^3, \frac{m/2^2}{4}) + c(n/8^2+m/2^2))&&\\
                 & + (T(n/8^3, \frac{\frac{m/2}{4}}{2}) + T(n/8^3, \frac{m/2}{4^2}) + c(n/8^2+\frac{m/2}{4}))&&\\
                 & + (T(n/8^3, \frac{n/4}{2^2}) + T(n/8^3, \frac{\frac{n/4}{2}}{4}) + c(n/8+\frac{n/4}{2}))&&\\
                 & + (T(n/8^3, \frac{n/4^2}{2}) + T(n/8^3, n/4^3) + c(n/8+n/4^2))&&\\
                 & + c(n+m)&&\\
           =   \ & (T(n/8^3, m/2^3) + T(n/8^3, m/2^4) + c(n/8^2+m/2^2))&&\\
                 & + (T(n/8^3, m/2^4) + T(n/8^3, m/2^5) + c(n/8^2+m/2^3))&&\\
                 & + (T(n/8^3, n/2^4) + T(n/8^3, n/2^5) + c(n/8^2+n/2^3))&&\\
                 & + (T(n/8^3, n/2^4) + T(n/8^3, n/2^6) + c(n/8^2+n/4^2))&&\\
                 & + c(n+m)&&\\
           \leq\ &\ldots&&\\
\end{flalign*}
\begin{flalign*}
    T(n,m) \leq\ & c(\sum_{i=1}^{k_1} 2*(n/8^i) + m/2^i + m/2^{i+1}&&\\
                 & + \sum_{i=1}^{k_2} 2*(n/8^i) + n/2^{2+i} + n/2^{2+i+1}&&\\
                 & + m+n)&&\\
           =   \ & c(2 n \sum_{i=1}^{k_1} 1/8^i + m (\sum_{i=1}^{k_1} 1/2^i +\sum_{i=1}^{k_1} 1/2^{i+1})&&\\
                 & 2 n \sum_{i=1}^{k_2} 1/8^i + n (\sum_{i=1}^{k_2} /2^{2+i} + \sum_{i=1}^{k_2} 1/2^{2+i+1})&&\\
                 & + m+n)&&\\
           \leq\ & c(4 n \sum_{i=0}^{\infty} 1/8^i &&\\
                 & + m (\sum_{i=0}^{\infty} 1/2^i +\sum_{i=0}^{\infty} 1/2^{i})&&\\
                 & + n/2 (\sum_{i=0}^{\infty} 1/2^{i} + \sum_{i=1}^{\infty} 1/2^{i})&&\\
                 & + m+n)&&\\
           =   \ & c(4 n (8/7)&&\\
                 & + m (2 + 2)&&\\
                 & + n/2 (2 + 2)&&\\
                 & + m+n)&&\\
           =   \ & c(n(3+32/7) + 5m) \in O(m+n)&&\\
\end{flalign*}

\end{document}
