%!TEX root = ./graph_reduction.tex
\subsection{Bor\r uvka-Phasen}

\subsubsection{Idee und Ablauf}
\label{sec:borIdea}

Bor\r uvka-Phasen beruhen auf der Erkenntnis, dass in einem beliebigen 
    ungerichteten Graphen $G$ f"ur jeden Knoten $v \in V$ die inzidente Kante mit 
    minimaler Gewichtung 
    $e_{v^{min}} := \{v, u\}, u \in adj(v): 
        \nexists e' = \{v, u'\}, u' \in adj(v): w(e') < w(e_{v^{min}})$
    im MST von $G$ enthalten ist.\\
Hierf"ur k"onnen wir einen beliebigen Knoten $v \in V_G$ und die zu $v$ inzidente
    Kante mit minimalem Gewicht $e_{min}$ in einem 
    zusammenh"angenden Graphen $G$ betrachten.\\
1. Fall $adj(v)$ enth"alt nur eine Kante oder mehrere inzidenten Kanten sind minimal. 
    Dann ist die minimale Kante zwangsweise im 
    MST enthalten oder ein MST mit gleichem Gewicht kann durch jede der minimalen
    Kanten konstruiert werden.\\
2. Fall $|adj(v)| > 1$ und nur eine Kante ist minimal. 
    Nehmen wir an, dass ein MST existiert, der $e_{min}$ 
    nicht enth"alt. 
    Der MST verbindet alle Knoten.
    Sei $e$ die zu $v$ inzidente Kante im MST.
    Nehmen wir $e_{min}$ in den MST auf, so erhalten wir einen Zyklus "uber $v$.
    Entferenen wir $e$, so ist der Zyklus aufgel"ost.
    Ferner wissen wir $w(e_{min}) < w(e)$, da $e_{min}$ und $e$ zu $v$ inzident
    sind und $e_{min}$ die einzige minimale, inzidente Kante ist.
    Folglich haben wir nicht nur den Zyklus wieder aufgel"ost und die 
    Baumeigenschaften gewahrt, sondern auch das gewicht des MST reduziert.
    Damit war der MST nicht minimal und auch kein MST.
    Aus dem Widerspruch folgt, dass die minimale, inzidente Kante eines jeden
    Knotens im MST von $G$ enthalten sein muss.\\
Des weiteren ist f"ur uns interessant, dass durch die markierten, minimalen
    Kanten kein Zyklus entsteht. 
    Damit ein Zyklus entsteht, m"usste ein Pfad geschlossen werden, 
    in dem mindestens eine zus"atzliche Kante hinzugenommen wird. 
    Dessen Kantengewicht m"usste kleiner als das der bereits inzidenten, 
    minimalen, markierten Kante sein.\\
Ferner bedeutet das, dass wir durch die zur Kontraktion markierten Kanten
    $E_{min}$ nach einer Bor\r uvka-Phase einen Wald $F$ in $G$ induzieren.
    Dies wird f"ur uns insbesondere dann interessant, wenn
    wir rekursiv einen Wald aufbauen m"ochten.\\
Eine Bor\r uvka-Phase hat also den folgenden Ablauf:
\begin{enumerate}
    \item Markiere inzidente Kanten $E_{min}$ mit minimaler Gewichtung
    \item Bestimme die verbundenen Komponenten in $G' = (V,E_{min})$
    \item Ersetze jede verbundene Komponente durch einen sie repr"asentierenden
          Knoten in $G'$ und erhalte den Graphen $G''$
    \item Entferne alle Selbstschleifen in $G''$
\end{enumerate}

\subsubsection{Reduktion der Knoten}
Es werden die durch $E_{min}$ verbundenen Komponenten auf je einen Knoten 
    reduziert.
    Wir interessieren uns f"ur den Worst-Case, also die maximale Anzahl an 
    Komponenten.
    Die kleinste verbundene Komponente w"are theoretisch ein
    einzelner Knoten ohne Kanten.
    An dieser Stelle w"urde unser Algorithmus aber terminieren.
    Die kleinste verbundene Komponente, die wir in einer Bor\r uvka-Phase 
    betrachten werden, besteht also aus zwei Knoten und einer Kante.
    Da $G$ zusammenh"angend ist, existieren damit maximal $n/2$ verbundene
    Komponenten.\\
Wir stellen fest, dass eine Bor\r uvka-Phase die Menge an Knoten in $G$ auf 
    maximal die H"alfte reduziert, n"amlich genau dann, wenn die markierten 
    Kanten ein perfektes Matching beim Bestimmen der verbundenen Komponenten
    ergeben.\\
