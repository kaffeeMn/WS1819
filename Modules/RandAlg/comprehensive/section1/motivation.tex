%!TEX root = ../compehension.tex
\section{Motivation}

\subsection{MST und MSF}
Der minimale Spannbaum, oder auch MST, stellt einen azyklischen 
    zusammenh"angenden Teilgraph aus $G$, der alle Knoten verbindet und
    dessen Summe von Kantengewichte $\sum_{e \in E_{MST}} w(e)$
    minimal ist dar.\\
Ist $G$ selbst nicht zusammenh"angend, so werden wir einen minimalen Spannwald
    als n"achst beste L"osung betrachten.
    Ein minimaler Spannwald besteht aus einer Sammlung von minimalen 
    Spannb"aumen f"ur die verbundenen Komponenten in $G$.\\

\subsection{Vorteile eines nichtdeterministischen Anzatz}

Wir werden in dieser Ausarbeitung zum Kapitel 10.3 aus dem Buch
    \cite{randAlg}
    einen
    randomisierten Ansatz f"ur einen Algorithmus, der einen minimalen Spannbaum
    in erwarteter Linearzeit aproximiert betrachten.\\
Das MST Problem ist in $P$, und es sind bereits 
    deterministische Algorithmen wie die von
    Prim, Kruskal, Bor\r uvka bekannt, 
    die mit einer Worst-Case Laufzeitschranke 
    von $O(m * log(n))$ das Problem l"osen.
    Zudem existiert der Algorithmus von Bernard Chazelle, f"ur den eine
    Worst-Case Laufzeitschranke von $O(m * log \beta(m,n))$ bekannt ist, wobei
    mit
    $\beta(m,n) = \{i | log^{(i)} n \leq m / n\}$ die inverse Ackermann Funktion
    verwendet wird. 
    $log^{(i)} n$ ist hierbei die $i$-te Anwendung von $log$ auf $n$.
    Dementsprechend steigt die Funktion $\beta$ so schwach, dass die
    aus ihr resultierenden Faktoren f"ur die Worst-Case Laufzeit hinsichtlich
    der Gr"o"se von Graphen in der Praxis als nahezu konstant angesehen werden 
    kann.\\
Wozu dient also ein nichtdeterministischer Algorithmus, der im Erwartungswert 
    Linearzeit ben"otigt?
    - Die Antwort auf diese Frage kann sowohl durch die komplexe Implementierung
    des Algorithmus von Chazelle, als auch der Stabilit"at, bzw. G"ute,
    des vorgestellten Algorithmus hinsichtlich seiner Laufzeit und Approximation,
    sowie nicht zuletzt durch das Betrachten des Algorithmus als akademisches
    Beispiel f"ur kreative Laufzeitverbesserungen begr"undet werden.\\
