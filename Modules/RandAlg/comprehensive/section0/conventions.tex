\section{Konventionen}
In dieser Ausarbeitung wird sich an den "ublichen Konventionen zur Notation
    von Variablen in Graphen orientiert. So definieren wir einen Graphen als
    $G = (V,E)$, mit der Anzahl von Knoten $n = |V|$ und Kanten $m = |E|$.
    Zu $G$ zugeh"orige Kanten und Knotenmengen k"onnen auch durch $V_G$, $E_G$
    als solche gekennzeichnet werden.
    F"ur jeden Knoten $v$ nehmen wir an, dass eine Adjazenzliste $adj(v)$ von
    adjazenten Knoten f"ur ihn vorhanden ist.
    Ferner betrachten wir ungerichtete gewichtete Graphen und bezeichnen 
    die Gewichtungsfunktion als $w: E \rightarrow \mathbb{R}$.\\
Des weiteren werden auch Wege, bzw. Pfade betrachten. 
    Dazu sei $P(\{v,u\})$ eine Funktion, die in einem Baum den Pfad 
    von $v$ nach $u$ als Knotenfolge angibt.
    Da es f"ur uns handlicher sein wird
    gleich mit den Kanten zu arbeiten, definieren wir $P_e$ als die Folge von
    Kanten auf dem Pfad.\\
