\section{Der MST-Algorithmus}

Nun da wir Verfahren zum reduzieren von Knoten und Kanten des Graphen $G$ kennen
    gelernt haben  k"onnen wir anfangen einen Algorithmus zu konstruieren.
    Die Ausgabe unseres Algorithmus sei ein Wald $F$ in $G$.\\
Die erste Entscheidung die wir treffen m"ussen, ob wir zuerst Knoten oder Kanten
    reduzieren. Da Bor\r uvka-Pahsen nur Kanten markieren, die im MST/MSF 
    enthalten sind fangen wur mit Bor\r uvka-Pahsen an und erhalten den Graphen 
    $G_1$.
    Dadurch veringern wir
    zun"achst noch deterministisch die Anzahl $F$-leichter Kanten in $G$.
    Anschlie"send konstruieren wir $G_2 = G(p)$, wir wissen also, dass $G$ 
    $\frac{n/2}{p}$ $F$-leichte Kanten nach diesen Verfahren enth"alt.\\
\begin{algorithm}
\KwData{Graph $G$}
\KwResult{Approximation eines MST/ MSF in $G$}
\begin{algorithmic}[1]
    \STATE $G_1, C$ $\leftarrow$\begin{tabular}[H]{l}
                                 3 Bor\r uvka-Phasen werden auf $G$ angewandt.\\
                                 Dabei wird der resultierende Graph und die zur\\
                                 Kokatenierung markierten Kanten zur"uck gegeben.
                                 \end{tabular}
    \STATE $G_2$ $\leftarrow$ $G_1(p=0,5)$
    \STATE $F_2$ $\leftarrow$ $MST(G_2)$
    \STATE $G_3$ $\leftarrow$ $(V_{G_1}, E_{G_1} - E_{F_2-heavy})$
    \STATE $F_3$ $\leftarrow$ $MST(G_3)$
    \RETURN $C \cup F_3$
\end{algorithmic}
\end{algorithm}

\subsection{Laufzeit}

$T(n,m)$ sei die erwartete Laufzeit des MST-Algorithmus f"ur einen Graphen $G$.
    Zeile 1 benutzt 3 Bor\r uvka-Phasen und  l"auft damit in $O(n+m)$. 
    $G_2 = G_1(p)$ aus Zeile 2 kann ebenfalls in $O(m + n)$ berechnet werden.
    $G_1$ hat nur noch $n/2^3 = n/8$ Knoten und f"ur $G_2$ hat damit 
    $|V_{G_2}| = n/8$ Knoten und im Erwartungswert $|E_{G_2}| = m/2$ Kanten.
    Folglich ben"otigt die Berechnung von $F_3$  die erwartete Laufzeit 
    $T(n/8,m/2)$.
    Die Berechnugn $F_2$-schwerer Kanten und die Konstruktion von $G_3$ mittels
    derer ben"otigt ebenfalls $O(m+n)$.
    Nach dem Lemma 10.19 ist die Anzahl von $F$-leichten Kanten in $G$ durch
    $n/p = 2n,$ mit $p = 0,5$ im Erwartungswert nach oben beschr"ankt. Da $G_1$
    aner nur $n/8$ Knoten hat, k"onnen wir die Anzahl an Kanten in $G_3$
    durch $2*n/8 = n/4$ im Erwartungswert absch"atzen.
    Damit bel"auft sich die erwartete Laufzeit der Berechnung von $F_3$ auf
    $T(n/2, n/4)$.
    Die Vereinigung von $C$ und $F$ in Zeile 6 ben"otigt $O(n)$.\\
Aus den Laufzeiten der Teilschritte folgt
    $T(n,m) \leq T(n/8, m/2) + T(n/8, n/4) + c(n+m)$
    \\
