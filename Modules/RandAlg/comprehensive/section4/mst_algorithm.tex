\section{Der MST-Algorithmus}

Nun da wir Verfahren zum reduzieren von Knoten und Kanten des Graphen $G$ kennen
    gelernt haben  k"onnen wir anfangen einen Algorithmus zu konstruieren,
    dessen Ausgabe ein Wald $F$ in $G$ ist.\\
Die erste Entscheidung die wir treffen m"ussen ist, ob wir zuerst Knoten oder Kanten
    reduzieren. \\
Da wir in \hyperref[sec:borIdea]{\textit{3.1.1}} gelernt haben, dass
    Bor\r uvka-Phasen nur Kanten markieren, die im MST/MSF 
    enthalten sind fangen wir mit Bor\r uvka-Phasen an und erhalten den Graphen 
    $G_1$.
    Dadurch veringern wir
    zun"achst noch deterministisch die Anzahl $F$-leichter Kanten in $G$.
    Sollten wir in in den Bor\r uvka-Phasen terminieren geben wir den Teilgraph
    mit allen markierten Kanten $C$ aus.
    Die Bor\r uvka-Phasen werden also auch das Abbruchkriterium unserer
    Rekursion sein.\\
Anschlie"send konstruieren wir $G_2 = G_1(p)$ und f"uhren einen rekursiven
    Aufruf auf diesem durch. 
    Wir wissen also durch \hyperref[sec:goodnessRand]{\textit{3.2.1}}, 
    dass $G$ im Erwartungswert
    $\frac{n/2}{p}$ $F_2$-leichte Kanten vor dem rekursiven Aufruf enth"alt.\\
Wir haben in \hyperref[sec:verification]{\textit{2.2.2}} einen 
    Verfikationsalgorithmus definiert, der auch $F_2$-schwere Kanten ausgeben
    kann.
    Da wir durch \hyperref[sec:fProof]{\textit{2.2.1}} wissen, dass $F_2$-schwere Kanten nicht im MST/MSF von $G$
    enthalten sein k"onnen, entfernen wir alle $F_2$-schwere Kanten $E_{F_2-heavy}$
    aus dem Graphen $G_1$,
    auf dem lediglich die Bor\r uvka-Phasen angewandt wurden und erhalten $G_3$.\\
F"uhren wir nun nocheinmal einen rekursiven Aufruf auf $G_3$ durch, 
    so erhalten wir 
    einen Wald $F_3$ mit intuitiv weniger $F_3$-schweren, als $F_2$-schweren Kanten in $G$.
    Ferner ist insbesondere in dichten Graphen zu erwarten, dass die Stichproben 
    auf $G_3$ nur unwahrscheinlich die Anzahl von Verbundenen Komponenten in 
    $G_3$ um ein wesentliches erh"ohen, bzw. Komponenten aus $G_3$ nicht mehr
    zusammen h"angen.
    Wir verursachen also keinen Schaden durch das Entfernen der $F_2$-schweren
    Kanten.\\
Nachdem wir den Wald $F_3$ bestimmt haben geben wir unsere, von den Bor\r uvka-Phasen
    markierten, Kanten in $C$ vereinigt mit $F_3$ aus.\\

\begin{algorithm}
\KwData{Graph $G$}
\KwResult{Approximation eines MST/ MSF in $G$}
\begin{algorithmic}[1]
    \STATE $G_1, C$ $\leftarrow$\begin{tabular}[H]{l}
                                 3 Bor\r uvka-Phasen werden auf $G$ angewandt.
                                 Dabei wird der resultierende Graph und\\
                                 ein Teilgraph $C$ mit den zur
                                 Kokatenierung markierten Kanten zur"uck gegeben.\\
                                 \textbf{Wenn} $G$ leer ist oder in den Bor\r uvka-Phasen
                                 terminiert wird geben wir $F=C$ aus.\\
                                 \end{tabular}
    \STATE $G_2$ $\leftarrow$ $G_1(p=0,5)$
    \STATE $F_2$ $\leftarrow$ $MST(G_2)$
    \STATE $G_3$ $\leftarrow$ $(V_{G_1}, E_{G_1} - E_{F_2-heavy})$
    \STATE $F_3$ $\leftarrow$ $MST(G_3)$
    \RETURN $F = C \cup F_3$
\end{algorithmic}
\end{algorithm}

\subsection{Laufzeit}

$T(n,m)$ sei die erwartete Laufzeit des MST-Algorithmus f"ur einen Graphen $G$.
    Zeile 1 benutzt 3 Bor\r uvka-Phasen und  l"auft damit in $O(n+m)$. 
    $G_2 = G_1(p)$ aus Zeile 2 kann ebenfalls in $O(m + n)$ berechnet werden.
    $G_1$ hat nur noch $n/2^3 = n/8$ Knoten und f"ur $G_2$ hat damit 
    $|V_{G_2}| = n/8$ Knoten und im Erwartungswert $|E_{G_2}| = m/2$ Kanten.
    Folglich ben"otigt die Berechnung von $F_2$  die erwartete Laufzeit 
    $T(n/8,m/2)$.
    Die Berechnugn $F_2$-schwerer Kanten und die Konstruktion von $G_3$ mittels
    derer ben"otigt ebenfalls $O(m+n)$.
    Nach dem Lemma 10.19 \cite{randAlg} ist die Anzahl von $F$-leichten Kanten in $G$ durch
    $n/p = 2n,$ mit $p = 0,5$ im Erwartungswert gegeben. Da $G_1$
    aber nur $n/8$ Knoten hat, k"onnen wir die Anzahl an Kanten in $G_3$
    durch $2 \cdot n/8 = n/4$ im Erwartungswert absch"atzen.
    Damit bel"auft sich die erwartete Laufzeit der Berechnung von $F_3$ auf
    $T(n/2, n/4)$.
    Die Vereinigung von $C$ und $F$ in Zeile 6 ben"otigt $O(n+m)$.\\
Aus den Laufzeiten der Teilschritte folgt
    $T(n,m) \leq T(n/8, m/2) + T(n/8, n/4) + c(n+m)$.
    Nach der Literatur kann der rekursive Anteil der Ungleichung durch $2c(n+m)$ 
    abgesch"atzt werden.
    %Ein Beweis, dass $T(n,m)$ tats"achlich in $O(n+m)$ enthalten ist befindet
    %sich im Anhang.\\
    \\
