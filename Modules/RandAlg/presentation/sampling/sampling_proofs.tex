\begin{frame}{Verschlechtern wir den MSF?}
    \begin{center}
    \begin{tabular}[h]{p{8cm}}
        \cellcolor{GreenTOL!40!}
        Lemma 10.19 (vereinfacht)\\
        \cellcolor{GreenTOL!20!}
        F"ur den MSF $F$ von $G(p)$, wobei $p \in (0,1]$ erwarten wir nicht mehr als
		$n/p$ $F$-leichte Kanten in $G$.\\[0.3cm]
    \end{tabular}
    \end{center}
    \gap
    Beweisidee:\\
    \begin{overlayarea}{\textwidth}{4cm}
        \ \\
        \begin{itemize}
            \only<2,3,4,5>{
                \item  Seien die Kanten von $G$ aufsteigend sortiert\\
            } 
            \only<2>{
                $$
                    e_1,        &\ldots     &, e_{m_G},\
                    w(e_1) \leq &\ldots     &\leq w(e_{m_G})
                $$
            }
            \only<3,4,5>{
                \item Sei $F = (V_G, \emptyset)$
            }
            \only<4,5>{
                \item Konstruiere $G(p)$ nach der Kantenreihenfolge
            }
            \only<5>{
                \item Ist die betrachtete Kante $F$-leicht wird sie in $F$ aufgenommen
            }
        \end{itemize}
    \end{overlayarea}
\end{frame}
\begin{frame}{Beweis: Phasen}
    \begin{overlayarea}{\textwidth}{1.5cm}
    \begin{itemize}
        \only<1>{
            \item Wann wird die n"achste Kante hinzugenommen?
        }\only<2,3,4,5,6>{
            \item Wann wird die n"achste $F$-leichte Kante hinzugenommen?
        }\only<4>{
            \item Wie oft \glqq w"urfeln\grqq?
        }\only<5,6>{
            \item Wie oft \glqq w"urfeln\grqq? \textcolor{orange}{- $1/p$}
        }\only<6>{
        \item Letzte Phase: $s \leq n-1, s \in \mathbb{N}$
        }
    \end{itemize}
    \end{overlayarea}
    \gap
    \begin{overlayarea}{\textwidth}{1cm}
    \only<3,4,5,6>{
        \begin{tabular}[h]{lll}
            $k$-te Phase $\overset{def}{=}$ &Zufallsexperimente 
                                                & ab $k-1$ Kanten in $F$,\\
                                                && bis $k$ Kanten in $F$\\
        \end{tabular}
    }
    \end{overlayarea}
\end{frame}
\begin{frame}{Beweis: Stochastische Dominanz}
    \begin{overlayarea}{\textwidth}{3cm}
    \begin{itemize}
        \only<1>{
            \item Differenz der Phasen $c = (n-1) - s$
        }\only<2,3,4,5,6>{
            \item Differenz der Phasen
                    $\textcolor{red}{c} = (n-1) - \textcolor{blue}{s}$
        }\only<3,4,5,6>{
            \item weitere $c$-Phasen, impliziert: erwartet mehr $F$-leichte Kanten 
                    in $G$\\
        }\only<4,5,6>{
            \item $n-1$ Phasen/ Erfolge
        }\only<6>{
            \item Erfolgswahrscheinlichkeit $p$, impiziert: negative 
            Binomialverteilung
        }
    \end{itemize}
    \end{overlayarea}
    \begin{overlayarea}{\textwidth}{2cm}
        \begin{center}
        \begin{tikzpicture}[minND/.style={circle,
                                          anchor=center,
                                          text width=2.5mm,
                                          inner sep=3pt,
                                          align=center}]
        \only<5,6>{
            \node[]                 (s1) at (0,1) {1/p};
            \node[]                 (s2) at (1,1) {$\ldots$};
            \node[]                 (s3) at (2,1) {1/p};
            \node[]                 (c1) at (3,1) {1/p};
            \node[]                 (c2) at (4,1) {$\ldots$};
            \node[]                 (c3) at (5,1) {1/p};
        }
        \only<2,3,4,5,6>{
            \node[minND, fill=blue] (s1) at (0,0) {};
            \node[]                 (s2) at (1,0) {$\ldots$};
            \node[minND, fill=blue] (s3) at (2,0) {};
            \node[minND, fill=red]  (c1) at (3,0) {};
            \node[]                 (c2) at (4,0) {$\ldots$};
            \node[minND, fill=red]  (c3) at (5,0) {};
            # braces
            \draw [ decoration={
                        brace,
                        mirror,
                        raise=0.5cm
                    },
                    decorate] 
                (s1.west) -- (s3.east) 
                node [pos=0.5,anchor=north,yshift=-0.55cm] {\textcolor{blue}{$s$}}; 
            \draw [ decoration={
                        brace,
                        mirror,
                        raise=0.5cm
                    },
                    decorate] 
                (c1.west) -- (c3.east) 
                node [pos=0.5,anchor=north,yshift=-0.55cm] {\textcolor{red}{$c$}}; 
            \draw [ decoration={
                        brace,
                        mirror,
                        raise=0.5cm
                    },
                    decorate] 
                (0,-0.5) -- (5,-0.5) 
                node [pos=0.5,anchor=north,yshift=-0.55cm] {$n-1$}; 
        }
            ;
        \end{tikzpicture}
        \end{center}
    \end{overlayarea}
\end{frame}

\begin{frame}{Beweis: Stochastische Dominanz}
	\vspace{-1cm}
	\begin{center}
    \begin{tikzpicture}[minND/.style={circle,
                                      anchor=center,
                                      text width=2.5mm,
                                      inner sep=3pt,
                                      align=center}]
        \node[]                 (ps1) at (0,1) {1/p};
        \node[]                 (ps2) at (1,1) {$\ldots$};
        \node[]                 (ps3) at (2,1) {1/p};
        \node[]                 (pc1) at (3,1) {1/p};
        \node[]                 (pc2) at (4,1) {$\ldots$};
        \node[]                 (pc3) at (5,1) {1/p};
        \node[minND, fill=blue] (s1) at (0,0) {};
        \node[]                 (s2) at (1,0) {$\ldots$};
        \node[minND, fill=blue] (s3) at (2,0) {};
        \node[minND, fill=red]  (c1) at (3,0) {};
        \node[]                 (c2) at (4,0) {$\ldots$};
        \node[minND, fill=red]  (c3) at (5,0) {};
        # braces
        \draw [ decoration={
                    brace,
                    mirror,
                    raise=0.5cm
                },
                decorate] 
            (s1.west) -- (s3.east) 
            node [pos=0.5,anchor=north,yshift=-0.55cm] {\textcolor{blue}{$s$}}; 
        \draw [ decoration={
                    brace,
                    mirror,
                    raise=0.5cm
                },
                decorate] 
            (c1.west) -- (c3.east) 
            node [pos=0.5,anchor=north,yshift=-0.55cm] {\textcolor{red}{$c$}}; 
        \draw [ decoration={
                        brace,
                        mirror,
                        raise=0.5cm
                    },
                    decorate] 
                (0,-0.5) -- (5,-0.5) 
                node [pos=0.5,anchor=north,yshift=-0.55cm] {$n-1$};
        ;
    \end{tikzpicture}
    \end{center}
	\vspace{-0.5cm}
    \begin{overlayarea}{\textwidth}{2cm}
    \ \\
    \only<2,3>{
        $X_{np} \overset{def}{=}$ negative Binomialverteilung, Parameter $n-1, p$\\
        $X_{sp} \overset{def}{=}$ negative Binomialverteilung, Parameter $s, p$\\
    }
    \only<3>{
        $X_{np}$ dominiert $X_{sp}$, mit: 
        \begin{center}
        \begin{tabular}[h]{ll}
            F"ur alle $z \in \mathbb{R}: & Pr[X_{np} > z] \geq Pr[X_{sp} > z]$\\
            oder auch:                   &$n/p > (n-1)/p = E[X_{np}] \geq E[X_{sp}] = s/p$\\
        \end{tabular}
        \end{center}
    }
    \end{overlayarea}
\end{frame}
