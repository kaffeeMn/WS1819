\begin{frame}{Verschlechtern wir den MSF?}
    \begin{center}
    \begin{tabular}[h]{l}
        \cellcolor{GreenTOL!40!}
        Theorem\\
        \cellcolor{GreenTOL!20!}
        F"ur den MSF $F$ von $G(p), p \in (0,1]$ gibt es $\frac{n}{p}$
        $F$ -leichte Kanten in $G$\\[0.3cm]
    \end{tabular}
    \end{center}
    \gap
    Beweisidee:\\
    \begin{overlayarea}{\textwidth}{4cm}
        \ \\
        \begin{itemize}
            \only<2,3,4,5>{
                \item  Seien die Kanten von $G$ aufsteigend sortiert\\
            } 
            \only<2>{
                $$
                    e_1,        &\ldots     &, e_{m_G},\
                    w(e_1) \leq &\ldots     &\leq w(e_{m_G})
                $$
            }
            \only<3,4,5>{
                \item Sei $F = (V_G, \emptyset)$
            }
            \only<4,5>{
                \item Konstruiere $G(p)$ nach der Kantenreihenfolge
            }
            \only<5>{
                \item Ist die betrachtete Kante $F$-leicht wird sie in $F$ aufgenommen
            }
        \end{itemize}
    \end{overlayarea}
\end{frame}
\begin{frame}{Beweis: Phasen}
    \begin{overlayarea}{\textwidth}{1.5cm}
    \begin{itemize}
        \only<1>{
            \item Wann wird die n"achste Kante hinzugenommen?
        }\only<2,3,4,5,6>{
            \item Wann wird die n"achste $F$-leichte Kante hinzugenommen?
        }\only<4>{
            \item Wie oft \glqq w"urfeln\grqq?
        }\only<5,6>{
            \item Wie oft \glqq w"urfeln\grqq? \textcolor{blue}{- $1/p$}
        }\only<6>{
        \item Anzahl an Phasen insgesamt: $s \leq n-1, s \in \mathbb{N}$
        }
    \end{itemize}
    \end{overlayarea}
    \gap
    \begin{overlayarea}{\textwidth}{1cm}
    \only<3,4,5,6>{
        \begin{tabular}[h]{lll}
            $k$-te Phase $\overset{def}{=}$ &Zufallsexperimente 
                                                & ab $k-1$ Kanten in $F$,\\
                                                && bis $k$ Kanten in $F$\\
        \end{tabular}
    }
    \end{overlayarea}
\end{frame}
\begin{frame}{Beweis: Stochastische Dominanz}
    \begin{overlayarea}{\textwidth}{3cm}
    \begin{itemize}
        \only<1,2,3,4,5>{
            \item Differenz der Kanten $c = (n-1) - s$
        }\only<2,3,4,5>{
            \item weitere $c$-Phasen $\Rightarrow$ erwartet mehr $F$-leichte Kanten 
                    in $F$\\
        }\only<3,4,5>{
            \item $s+c = n-1$ Phasen, bzw. $n$ Erfolge
        }\only<4,5>{
            \item Erfolgswahrscheinlichkeit: $p$ $\Rightarrow$ negative 
            Binomialverteilung
        }
    \end{itemize}
    \end{overlayarea}
    \begin{overlayarea}{\textwidth}{2cm}
    \ \\
    \only<5>{
        $X_{np} \overset{def}{=}$ negative Binomialverteilung, Parameter $n, p$\\
        $X_{sp} \overset{def}{=}$ negative Binomialverteilung, Parameter $s, p$\\
        $X_{np}$ dominiert $X_{sp}$, mit: $$
            n/p = E[X_{np}] \geq E[X_{sp}] = s/p
            $$
    }
    \end{overlayarea}
\end{frame}
