\section{Kennzahlen}
\subsection{Lage}
Uns interessiert bei Daten insbesondere die Verteilung derer.
    Wir möchten also mit möglichst wenigen und einfachen Mitteln
    erfassen wie Streuung/ Mittel-/ Medianwerte sich verhalten.\\
\\
\subsection{Allgemeine Definitionen}
\subsubsection{Mittel}
Arithm. Mittel:\\
$$
    \overset{-}{x} = \frac{1}{n} \sum_{i=1}^n x_i
$$
Median:\\
$$
    med_x := \begin{cases}
        x_(\frac{n+1}{2})                               &, \text{n ungerade}\\
        \frac{x_(\frac{n}{2}) + x(\frac{n}{2} + 1)}{2}  &, \text{sonst}
    \end{cases}
$$
Modus: Häufigster Wert.\\
\subsubsection{Quantile}
Ein $p$-Quantil, $p \in [0,1]$ ist eine Zahl, für die $100 \cdot p \%$ 
    kleiner-gleich sind und $100 \cdot (1-p) \%$ der Werte größer-gleich\\
$$ 
    Q_p := \begin{cases}
        x_{(j)}                             &,\text{$np$ nicht ganzzahlig,
                                                    $j := \lceil np \rceil$}\\
        \frac{x_{(j)} + x_{(j+2)}}{2}       &,\text{sonst,
                                                    $j := np$}
    \end{cases}
$$
\subsubsection{Abweichungen}
absolute Abweichung:\\
$$
    \Delta_a(x) = \sum_{i=1}^N |x_i - x|
$$
Quadratische Abweichung:\\
$$
    \Delta(x) = \sum_{i=1}^N (x_i - x)^2
$$
\\
Hierbei ist insbesondere interassant, dass $\Delata(x)$ für $x = \overset{-}{x}$
    minimal ist.\\

\subsection{Streuung}
Wir haben bereits Maße für die Lage kennengelernt, nun wollen wir wissen wie stark
    unsere Daten um die Lage varieren.\\
\subsection{allgemeine Definitionen}
\subsubsection{Abstände und Varianz um die Lage}
Varianz:\\
$$
    var_x = s_x^2 := \sum_{i=1}^n \frac{(x_i - \overset{-}{x})^2}{n-1}
$$
Die Standardabweichung ist offensichtlich die Wurzel der Varianz $s_x := var_x$\\
\\
Interquatilsabstand:\\
$$
    qd_x := q^4 - q_4
$$
\\
range/ Spannweite:\\
$$
    R_x := max(x) - min(x) = x(n) - x(1)
$$
Frage an Daniel: muss es nicht eigentlich $X$ anstatt $x$ sein?
\subsubsection{Kennzahlen der Streuung}
Variantionskoeffizient/ relative Standardabweichung:\\
$$
    v_x := \frac{s_x}{\overset{-}{x}
$$
Mittlere absolute Medianabweichung MD:\\
$$
    md_x := \frac{1}[n} \sum_{i=1}^{n} |x_i - med_x|
$$
mediane absolute Medianabweichung MAD:\\
$$
    md_x := med(|x_i - med_x|)
$$
