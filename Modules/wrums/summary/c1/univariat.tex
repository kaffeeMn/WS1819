\section{Univariate Daten}

\subsection{Skalentypen}

Daten können zu verschiedenen Skalenniveaus, bzw. Typen klassifiziert werden.
    Je nach Skalentyp sind dann bestimmte Herangehensweisen hinsichtlich der
    Analyse/ Aufbereitung der Daten sinnvoll.\\

\begin{tabular}{llll}
    nominal             & ordinal            & intervall         & verhältniss\\
    Klassen-label       & Größenordnung      & Differenzen       & Verhältnisse\\
\end{tabular}

\subsection{Generelle Definitionen}
\subsubsection{Häufigkeiten}
Absolute Häufigkeit:\\
$$
    N_j = N[x(j)] = \sum_{i=1}^N d_i(j), d_i(j) := I_{x(e_i) = x(j)}
$$
\\
Relative Häufigkeit:\\
$$
    f_j = \frac{N_j}{N}
$$
\\
Population:\\
$$
    M_N = \{e_1, \ldots, e_N\}
$$
\\
Quantitatives Merkmal $X$ mit Ausprägung $x \in W_X$\\
Wertebereich $W_X$ von $X$:\\
$$
    W_X = \{x(j) | j=1, \ldots, J\}
$$
\\
Urliste:
$$
    D_N = \{x_n | n=1, \ldots, N\}
$$
Rangliste:\\
$$
    R = x_{(1)}, \ldots, x_{(N)} 
$$
, mit  $x_{(1)} \leq \ldots \leq x_{(N)}$\\
